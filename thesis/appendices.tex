\chapter{APPENDICES}
%TODO chapter page in pdf contents is 1 only for this section
\section*{APPENDIX A.1}
\vglue6pt
% For Appendix A.1
% Format the equation environment
\renewcommand{\theequation}{A.1.\arabic{equation}}
% Reset the counter
\setcounter{equation}{0}
\section{Installation of Software}

As the LLVM codebase is large and has many options while building from source, finding the right options that our computers can handle easily was both essential to get started and critical as it decides the time it takes to see a change in code to get compiled. For this purpose, we accumulated the commands and created a tutorial that we can use in the future.

\begin{lstlisting}[language=Bash, caption={Clone Repository and Install Necessary Packages}]
git clone https://github.com/llvm/llvm-project
cd llvm-project
mkdir build
cd build	
sudo apt install cmake, ninja-build, clang, lld	
\end{lstlisting}

\begin{lstlisting}[language=Bash, caption={CMake Configuration We Used}]
cmake -S ../llvm . -G Ninja -DCMAKE_BUILD_TYPE="Debug"  \
-DBUILD_SHARED_LIBS=True -DLLVM_USE_SPLIT_DWARF=True  \
-DLLVM_BUILD_TESTS=True   -DCMAKE_C_COMPILER=clang \
-DCMAKE_CXX_COMPILER=clang++ -DLLVM_TARGETS_TO_BUILD="all" \
-DLLVM_EXPERIMENTAL_TARGETS_TO_BUILD="RISCV" -DLLVM_ENABLE_LLD=ON	
\end{lstlisting}
With this command, we are choosing the type as debug. Shared\_libs=TRUE causes all libraries to be built shared instead of static libraries. ..SPLIT\_DWARF is set to True to minimize memory usage at link time. We want to use clang as the C compiler. Therefore, it is specified in the command as DCMAKE\_C\_COMPILER=clang. In addition to that, we want to use lld as the linker instead of gold, so we specify that as well.
This configuration is the most efficient in terms of memory and disk usage among our previous attempts at building LLVM from source.
\noindent
\begin{minipage}[t]{\linewidth}
\begin{lstlisting}[language=Bash, caption={To build from scratch or to rebuild files with change, automatically}]
ninja
\end{lstlisting}
\end{minipage}
% \hspace{2em}

\begin{minipage}[t]{\linewidth}
\begin{lstlisting}[language=Bash, caption={To build llc only which is the binary we modify}]
ninja llc	
\end{lstlisting}
\end{minipage}



While running ninja, CPU and ram usage significantly increases. All available cores are used capacity. This may prevent doing other tasks while running ninja. In order to prevent this one may opt to use the following command instead. It allows you to choose how many cores are going to be utilized.
\begin{lstlisting}[language=Bash]
	ninja llc -j<number of cores to use>
\end{lstlisting}

\newpage
\section*{APPENDIX A.2}
\vglue6pt
% For Appendix A.2
% Format the equation environment
\renewcommand{\theequation}{A.2.\arabic{equation}}
% Reset the counter
\setcounter{equation}{0}
\section{Unoptimized S-box IR Code}
The output of the unoptimized S-box function is provided below. The C code used to produce this LLVM IR is in Code \ref{lst:sbox-c}

\lstinputlisting[language=llvm,style=nasm, label={lst:unopt-sbox}]{../s-box/opt/unoptimized.ll}

\newpage

\section*{APPENDIX A.3}
\vglue6pt
\renewcommand{\theequation}{A.2.\arabic{equation}}
\setcounter{equation}{0}

\section{Creating Assembly File From C File}

LLVM consists of many flexible libraries which allows user to use different libraries with preffered  options. To create RISC-V assembly from c code, Clang and LLC are used with the following commands. \\

Clang is the C compiler front-end which is mainly used with LLVM back-end. Clang is used in this project to produce LLVM intermediate representation code. Following command produces a .ll file in the current directory. 

\begin{lstlisting}[language=Bash]
-clang -S -target riscv32-linux-gnu -emit-llvm foo.c
\end{lstlisting}

-S option provides to run only preprocess and compilation steps. \\
-target option specifies the 32-bit RISC-V target architecture. \\
-emit-llvm is for targeting the LLVM back-end. \\

LLC is the LLVM compiler back-end which converts LLVM intermediate representation (IR) into native machine code for a specific target architecture. Following command produces a .s file for RISC-V architecture in the current directory. 

\begin{lstlisting}[language=Bash]
llvm-project/build/bin/llc -debug-only=isel -view-isel-dags -mtriple=riscv32 lxr.ll
\end{lstlisting}
%TODO lets set standard bin in this thesis, similar todo exists 

-debug-only=isel option gives the debug informations during the DAG lowering process.\\
-view-isel-dags option prints the DAG image of the IR code.\\
-view-sched-dags option can be used instead of -view-isel-dags, if the non scheduled DAG wants to be shown.\\
-mtriple=riscv32 defines the 32-bits RISC-V target architecture.\\
\newpage



\section*{APPENDIX A.4}
\vglue6pt
% For Appendix A.2
% Format the equation environment
\renewcommand{\theequation}{A.2.\arabic{equation}}
% Reset the counter
\setcounter{equation}{0}
\section{Adding the Crypt extension to the LLVM}
To add our extension to LLVM, a new file named RISCVInstrInfoCrypt.td should be created in ../llvm-project/llvm/lib/Target path and code \ref{lst:crypt_file} should be pasted in this file.

\begin{lstlisting}[caption={RISCVInstrInfoCrypt.td file},label={lst:crypt_file}]
	// Instruction class templates
	

	let hasSideEffects = 0, mayLoad = 0, mayStore = 0 in
	class ALU_rrr<bits<2> funct2, bits<3> funct3, string opcodestr,
				 bit Commutable = 0>
		: RVInstR4<funct2, funct3, OPC_OP, (outs GPR:$rd), (ins GPR:$rs1, GPR:$rs2, GPR:$rs3),
				  opcodestr, "$rd, $rs1, $rs2 ,$rs3"> {
	  let isCommutable = Commutable;
	}
	
	// Instructions
	
	
	def MLA     : ALU_rrr<0b10, 0b100, "mac">,
				  Sched<[WriteIMul, ReadIMul, ReadIMul]>;
	
	
	def NAXOR     : ALU_rrr<0b11, 0b100, "naxor">,
				  Sched<[WriteIMul, ReadIMul, ReadIMul]>;
	
	
	def SHLXOR  : ALU_rr<0b0011000, 0b111, "shlxor">,
			   Sched<[WriteIALU, ReadIALU, ReadIALU]>;
	
	
	let mayLoad = 1 in{
	def LXR  : ALU_rr<0b0011011, 0b101, "lxr">,
			   Sched<[WriteIALU, ReadIALU, ReadIALU]>;
	}
	
	// Instruction infos
	
	
	def : Pat<  (add (mul GPR:$src1, GPR:$src2), GPR:$src3),
				(MLA GPR:$src1, GPR:$src2, GPR:$src3)>;
	
	def : Pat<  (xor (and (not GPR:$src1), GPR:$src2), GPR:$src3),
				(NAXOR GPR:$src1, GPR:$src2, GPR:$src3)>;
	
	def : Pat<  (xor (shl GPR:$src1, (i32 1)), GPR:$src2),
				(SHLXOR GPR:$src1, GPR:$src2)>;
	
	def : Pat<  (xor (load GPR:$rs1),(load GPR:$rs2)),
				(LXR GPR:$rs1,GPR:$rs2)>;
	
\end{lstlisting}

After the file is created properly, code \ref{lst:includecrypt} should be added at the end of the ../llvm-project/llvm/lib/Target/RISCV/RISCVInstrInfo.td file.
\begin{lstlisting}[caption={Include line},label={lst:includecrypt}]
include "RISCVInstrInfoCrypt.td"
\end{lstlisting}
New extension will be ready to use after building the LLVM.
\newpage
