%============================================== NEW INPUTS ==============================================
\usepackage[T1]{fontenc} 				% For Turkish characters
\usepackage[utf8]{inputenc}  			% For Turkish characters
\usepackage[open,openlevel=1]{bookmark} % It is used to have bookmarks in the PDF file created
% Abbreviations & Nomenclature
%\usepackage[refpage]{nomencl}
%\makenomenclature
%\usepackage{textcomp}
%\usepackage{array}
%\usepackage{lscape}
\usepackage{listings} % Required for insertion of code
\usepackage{xcolor}
%\usepackage[colorlinks=true,linkcolor=blue,urlcolor=black,bookmarksopen=true]{hyperref}
%\usepackage{biblatex}

%====================================== ITU NEW PACKAGES INCLUDED ======================================
\usepackage{color}
\usepackage{times}
\usepackage{amssymb,amsmath,mathptmx,amsbsy,bm}
\usepackage{caption}            
%\usepackage{floatflt}
%\usepackage[dvipdfm]{graphicx}
\usepackage{graphics}
\usepackage{wrapfig}
\usepackage{epsfig}
\usepackage{enumerate}
\usepackage{rotating}
\usepackage{multirow}					% Multirow in tables
%\usepackage{subfigure} 				% This is obsolete, therefore use subcaption package instead - SBÖ
\usepackage{colortbl}
\usepackage{pstricks}
\usepackage{pst-plot}
\usepackage{cite}
\usepackage{latexsym}
%\usepackage{subeqn}
%\usepackage{hyperref}
\usepackage{hyperref}\hypersetup{hidelinks} % The line is added for hiding the links in document.
%\usepackage{url}
\usepackage{fixltx2e} % Bu paketi sembollerde text ler için subscript yazmakta yardımcı olması için ekliyoruz.
%\usepackage{ulem} 			% That does destroy Dedication and Bibliography (Use the below version xpatch) - SBÖ 
\usepackage{xpatch} 		% Added for TOC dot flush to the page number - SBÖ
\usepackage[normalem]{ulem} % For special underline tricks at the top of kapak pages - SBÖ 
\usepackage[bottom,multiple]{footmisc} 	% Add hang to align to the left - SBÖ (bottom,flushmargin)
%\usepackage{fnpos} 					% Another footnote position package - SBÖ
\setlength{\skip\footins}{1cm} 			% Placement of the last text from the top of the Footnote - SBÖ
%\setlength{\skip\footins}{1\baselineskip}
\setlength\footnotemargin{.35em} 		% Footnote indentation the first line from left margin - SBÖ
\addtolength{\footnotesep}{1mm} 		% Distance change to 1 mm line with footnote text at the bottom - SBÖ
%\setlength{\footnotesep}{.5\baselineskip}
\usepackage{enumitem} 					% Used for bullets in the resume to flush left as in the word template - SBÖ
\renewcommand\labelitemi{\normalsize$\bullet$} % Set the bullet size similar to the word template - SBÖ
%\makeFNbottom 							% Used with fnpos package - SBÖ
\usepackage{pdfpages} 					% http://ctan.org/pkg/pdfpages - SBÖ
%\usepackage{fancyhdr} 					% http://ctan.org/pkg/fancyhdr - SBÖ
%\usepackage{geometry}
\usepackage{tikzpagenodes} 				% For landscape page numbering - SBÖ
%\usepackage{setspace} 					% Provides support for setting the spacing between lines - SBÖ
%\usepackage{showframe} 				% http://ctan.org/pkg/showframe - To show the margins in a frame on pages - SBÖ
\usepackage{etoolbox}					% http://ctan.org/pkg/etoolbox - For removing default 50pt TOx stuffs from top - SBÖ

\usepackage{ragged2e}
\usepackage{multicol}
\usepackage{llvm/lang}  % include custom language for LLVM IR.
\usepackage{riscv-tex/lang}  % include custom language for RISCV assembly.
\usepackage{riscv-tex/style} % include custom style for NASM assembly.
\usepackage{csquotes}

\renewcommand{\lstlistingname}{Code}% Listing -> Algorithm
\renewcommand{\lstlistlistingname}{List of \lstlistingname s}% List of Listings -> List of Algorithms
\definecolor{codegreen}{rgb}{0,0.6,0}
\definecolor{codegray}{rgb}{0.5,0.5,0.5}
\definecolor{codepurple}{rgb}{0.58,0,0.82}
\definecolor{backcolour}{rgb}{0.95,0.95,0.92}
\lstdefinestyle{mystyle}{
    %TODO this is a terrible default style
    backgroundcolor=\color{backcolour},
    commentstyle=\color{codegreen},
    keywordstyle=\color{magenta},
    numberstyle=\tiny\color{codegray},
    stringstyle=\color{codepurple},
    basicstyle=\ttfamily\footnotesize,
    breakatwhitespace=false,
    breaklines=true,
    captionpos=b,
    keepspaces=true,
    numbers=left,
    numbersep=5pt,
    showspaces=false,
    showstringspaces=false,
    showtabs=false,
    tabsize=2
}
\lstset{style=mystyle}

\makeatletter 															% Used with etoolbox - SBÖ
\patchcmd{\@makechapterhead}{\vspace*{50\p@}}{}{}{}						% Removes space above \chapter head
\patchcmd{\@makeschapterhead}{\vspace*{50\p@}}{\vspace*{21.5mm}}{}{}	% Removes space above \chapter* head and add 21.5mm
\makeatother
\usepackage{longtable} 	% Include long tables in the text spreading more than one page - SBÖ
\usepackage{hhline} 	% If desired to eliminate hline in the tables - SBÖ
\usepackage{siunitx}
%\usepackage{subcaption} % Make subfigure as Figure 1.1a style - SBÖ
\usepackage[list=true,listformat=simple,position=below]{subcaption} % Make subfigure as Figure 1.1a style - SBÖ
% Subfigure caption settings - SBÖ
\DeclareCaptionLabelFormat{subfig}{\normalsize\figurename #1~\arabic{chapter}.\arabic{chapter}\alph{subfigure} :}
%\DeclareCaptionListFormat{subfigure}{\arabic{chapter}.\arabic{chapter}\alph{subfigure}}
\captionsetup[subfigure]{labelformat=subfig, size=normalsize}

% Bold Equation Number, Unbold Reference
\makeatletter
\def\tagform@#1{\maketag@@@{\ignorespaces#1\unskip\@@italiccorr}} % All bold in eq. number incl. parentheses 
%\renewcommand{\eqref}[1]{\textup{{\normalfont(\ref{#1}}\normalfont)}}
\renewcommand{\eqref}[1]{\textup{\bf(\ref{#1})}} % All bold in eq. referencing with parentheses - SBÖ
\makeatother

%\renewcommand{\theequation}{{\bf\arabic{chapter}.\arabic{equation}}} % Bold equation, unbold reference - SBÖ


%================================ MAKE THE CODE SHORTER - SOME EXAMPLES ================================
\def\be{\begin{equation}} %
\def\ee{\end{equation}}%
\def\beq{\begin{eqnarray}}%
\def\eeq{\end{eqnarray}}%
\def\bse{\begin{subequations}}%
\def\ese{\end{subequations}}%
\def\nonu{\nonumber}%
\def\psibar{\overline{\psi}} %
\def\Delslash{\partial\!\!\!\!\!/}%
\def\Gslash{G\!\!\!\!\!/}%
\def\Lmbdslash{\lambda\!\!\!\!\!/}%
\def\Jslash{J\!\!\!\!\!/}%
\def\pslash{p\!\!\!\!/}%
\def\qslash{q\!\!\!\!/}%
\def\kslash{k\!\!\!\!/}%
\def\[{\left[}
\def\]{\right]}
\def\({\left(}
\def\){\right)}

\def\Lag{{\cal L}}    % 
\def\D{{\cal D}}      % 
\def\H{{\cal H}}      % 
\def\Z{{\cal Z}}      % 
\def\S{{\textbf{S}}}  % 
\def\d{{\rm d}}%
\def\dt{{\rm{dt}}}%
\def\Tr{{\rm{Tr}}}%
\def\gm{\gamma^{\mu}}
\def\ga{\gamma^{\alpha}}
\def\gb{\gamma^{\beta}}
\def\gn{\gamma^{\nu}}
\def\gs{\gamma^{\sigma}}
\def\gl{\gamma^{\lambda}}
\def\gr{\gamma^{\rho}}
\def\gd{\gamma^{\delta}}


