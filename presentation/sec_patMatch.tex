
\section{Pattern Matching}
\subsection{TableGen Pattern}
\begin{frame}[fragile]{TableGen Patterns}
    RISC-V TableGen classes can be used to declare any instruction in a more structured way.
    \par
    The DAG pattern of the instruction can be declared in a Pattern declaration of TableGen. For the MLA (Multiply and Add) instruction, it is:

\begin{lstlisting}
def : Pat< (add (mul GPR:$src1, GPR:$src2), GPR:$src3),
(MLA GPR:$src1, GPR:$src2, GPR:$src3)>;
\end{lstlisting}
LLVM IR assembly instructions and intrinsic functions are combined in DAG structure. 
\end{frame}

\subsection{Pattern Matching with C++ in SelectionDAG}
\begin{frame}[fragile]{Limitations of TableGen}
\begin{itemize}
    \item TableGen patterns can only match tree structured patterns.
    \item Cannot match pattern with input relations. All inputs should be independent of each other.
    \item Limited by the language compared to C++.
\end{itemize}

\lstinputlisting[caption={Minimal IR Example of xor(load(x), load(x + 16)},linerange={1-5} ,label={lst:sbox-xor},language=llvm,style=nasm]{../s-box/keccakO3.ll}
\end{frame}

\begin{frame}{Pattern Matching with C++ in SelectionDAG}
\begin{enumerate}
    \item Create a record declaration in TableGen to provide the Assembler support, ignoring the Pattern declaration.
    \item Observe the DAG in the debug output or dot file and locate the root of it. 
    \item Add a function in RISCVISelDAGToDAG.cpp file in the root instruction case. 
    \item Implement pattern matching and replacement with SelectionDAG node.
\end{enumerate}
\end{frame}

\begin{frame}{Example with C++}{LXR Instruction}
\begin{itemize}
    \item Our example instruction LXR, loads two elements from an array and XOR's them.
    \item lxr rd, rs1, rs2 = xor(load(x), load(x + 16))

    \item Because the instruction loads from an array, the address values are relative to each other. 
    \item So TableGen cannot be used for pattern matching.
    \item But TableGen can always be used for instruction encoding thus Assembler support.
\end{itemize}
\end{frame}

\begin{frame}[fragile]{Instruction Encoding of LXR in TableGen}{1.Create a Record}
The input and outputs are registers so TableGen class can used for R-type instructions.

We can override funct7 and funct3 for our custom instruction encoding.
\begin{lstlisting}
let mayLoad = 1 in{
def LXR : ALU_rr<0b0011011, 0b101, "lxr">,
Sched<[WriteIALU, ReadIALU, ReadIALU]>;
}
\end{lstlisting}
llvm-mc assembler can be used for emitting binary from Assembly string.
\end{frame}

\begin{frame}[fragile]{DAG of LXR Pattern}{2.Observe the DAG}
The LXR targeting pattern is produced and dumped in SelectionDAG.
\lstinputlisting[caption={The corresponding Optimized and Legalized DAG}, linerange={3-3,4-4,8-9,13-13,16-16} ,label={lst:sbox-xor-dag},language=llvm,style=nasm]{../s-box/opt-lowered-dag.td}
Instruction Selection works on optimized, legalized DAG in SelectionDAG.
\end{frame}

\begin{frame}[fragile]{Modify RISCVISelDAGToDAG.cpp}{3.Add function call}
    RISCVISelDAGToDAG.cpp contains DAG to DAG transformations in C++.
\lstinputlisting[caption={Introduction of New Function for Pattern Matching in C++},label={lst:sbox-iseldagtodag},language=C++]{../s-box/custom_c++/iseldagtodagPlace.cpp}
\end{frame}


\begin{frame}[fragile]{Pattern Matching Function in C++}{4.Pattern Match Logic}
\lstinputlisting[caption={C++ logic, Part 1}, label={lst:sbox-cpp}, linerange={1-17} ,language=C++]{../s-box/custom_c++/xor_loads.cpp}
\end{frame}

\begin{frame}[fragile]{Pattern Matching Function in C++}{4.Pattern Match Logic}
\lstinputlisting[caption={C++ logic, Part 2}, label={lst:sbox-cpp2}, linerange={19-33} ,language=C++]{../s-box/custom_c++/xor_loads.cpp}
\end{frame}

\begin{frame}{Pattern Checks in the Function}
Check if..
\begin{enumerate}
    \item .. the operands are both Load instructions.
    \item .. the Second Load instruction has an Add instruction in its second operand
    \item .. the base offset of the first load and the first addendum of the second load are the same, as they should point to the beginning of the struct.
    \item .. the second addendum is a constant.
    \item .. the unsigned value of constant second addendum is equal to 16.
\end{enumerate}
\end{frame}

\subsection{Pattern Matching with C++ in LLVM IR}
\begin{frame}{Problems with SelectionDAG}
    \begin{itemize}
        \item Developed since 2005s
        \item Monolithic, hard to test passes
        \item Works on basic blocks, cannot access DAG of two functions at the same time.
        \item Affects compile time significantly
        \item Documentation is scarce
        \item Hard to debug as the DAG is built in memory
    \end{itemize}
    The new alternative is GlobalISel which addresses these problems but is still under development for RISC-V target.

    GlobalISel is default for AArch64 target.
\end{frame}

\begin{frame}{LLVM IR}
It is possible to lift pattern matching from the backend to the LLVM IR level. A procedure similar to the backend 
\end{frame}
