\section{RISC-V Description in SelectionDAG}
\subsection{TableGen Record Declaration}
\begin{frame}{TableGen}
    \begin{itemize}
        \item Domain-specific language used in LLVM backend side to generate CPP header files.
        \item
        Removes redundancy of instruction declaration code
        \item
        Classes are used to convey common information and get inherited to records
        \item
        Declarative instead of Imperative
    \end{itemize}
\end{frame}

\begin{frame}[fragile]{RISC-V TableGen Classes}
Target Independent Instruction Classes
    \begin{itemize}
        \item \textbf{InstructionEncoding}, decoder method, size of instruction
        \item
        \textbf{Instruction}, input and output DAGs
    \end{itemize}
RISC-V Instruction Classes Inherited to declare 'XOR' Instruction
    \begin{itemize}
        \item \textbf{RVInst}, universal bit patterns of RISC-V
        \item
        \textbf{RVInstR}, R type instruction
        \item
        \textbf{ALU\_rr}, features like Commutability declared
    \end{itemize}
\begin{lstlisting}
def XOR  : ALU_rr<0b0000000, 0b100, "xor", /*Commutable*/1>,
           Sched<[WriteIALU, ReadIALU, ReadIALU]>;
\end{lstlisting}
\end{frame}


\subsection{TableGen Pattern Matching}
\begin{frame}[fragile]{TableGen Patterns}
    RISC-V TableGen classes can be used to declare any instruction in a more structured way.
    \par
    The DAG pattern of the instruction can be declared in a Pattern declaration of TableGen. For the MLA (Multiply and Add) instruction, it is:

\begin{lstlisting}
def : Pat< (add (mul GPR:$src1, GPR:$src2), GPR:$src3),
(MLA GPR:$src1, GPR:$src2, GPR:$src3)>;
\end{lstlisting}
LLVM IR assembly instructions and intrinsic functions are combined in DAG structure. 
\end{frame}
