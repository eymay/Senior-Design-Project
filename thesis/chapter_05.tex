%%%%%%%%%%%%%%%%%%%%%%%%%%%%%%%%%%%%%%%%%%%%%%%%%%%%%%%%%%%%%%%%%
% !TEX root = interimreport.tex
\clearpage
\chapter{ADDING CUSTOM INSTRUCTIONS}\label{ch:custom_instr}
%%%%%%%%%%%%%%%%%%%%%%%%%%%%%%%%%%%%%%%%%%%%%%%%%%%%%%%%%%%%%%%%%
The instruction selection system we focused on at the back-end of the LLVM compiler is SelectionDAG among FastISel and GlobalIsel. SelectionDAG is the most mature Instruction Selection framework with more target support. However, in the near future, it is worth considering GlobalISel as it is developed recently as an alternative to SelectionDAG. The reasons to replace it are to make it faster, smaller, and more open to low-level optimizations.
\section{TableGen Reference}
TableGen is a domain-specific language used in LLVM backend side to generate CPP header files. The purpose it serves is that it reduces redundancy of instruction declaration code which can be common to numerous architectures with minor differences. To maintain and scale the framework the minor differences are implemented level by level at a series of inheritance operations between TableGen classes. 

\par
 LLVM Static Compiler, LLC, is responsible for converting LLVM IR to Assembly code. To add new instructions LLC is recompiled and its internals change. While the compilation operation of the LLC program, TableGen records are created which declare every instruction’s encoding and describe its features. Referring to the records, directed acyclic graphs (DAG) are used in the process of instruction selection. DAG is a graph structure which has no cycles and has directions on the edges. Operations or functions are represented as nodes in the DAG. They are critical parts of declaring the logic or pattern of the new instruction. 
\par

The operations represented on the DAG can be LLVM intrinsics as well as instructions. LLVM instructions resemble conventional assembly instructions, in contrast, LLVM intrinsics have higher level abstraction depending on their functionality. Their instruction generation may vary depending on the target hardware. It is possible to define a new complicated instruction either by combining simple LLVM instructions and higher level intrinsics in DAG level or by creating a new LLVM intrinsic which gets created at the Intermediate Level of the compilation process.

\section{RISC-V TableGen Classes}
The most general instruction class used for every target architecture is the “InstructionEncoding” TableGen class defined in llvm/include/llvm/Target/Target.td. This class holds the decoder method and size of instruction in addition to minor variables. It gets inherited to the generic “Instruction” class which is defined in the same class. This class holds input and output DAGs and information which is useful to the compiler and is generalizable to all architectures.
\par

The general class gets inherited to every target-specific class. In RISC-V’s case, the next stop of the instruction is the “RVInst” class which inherits from the general “Instruction” class and it resides in llvm/lib/Target/RISCV/RISCVInstrFormats.td TableGen file. It defines the general bit patterns of RISC-V instructions. For example, the opcode being the first 7 bits. It defines additional information like the assembly string pattern. This general class is inherited by every type of instruction of R, I, S, B, U, and J types. As a simple example, XOR instruction can be traced. As XOR is an R type, a register-register instruction, it continues its inheritance journey from “RVInstR”. It is common to R type instructions to have funct7, rs2, rs1, funct3, and rd format ordered from MSB to LSB. These variables are assigned corresponding bit fields in the class. 
\par

The RISC-V formats mentioned are included in the llvm/lib/Target/RISCV/RISCVInstrInfo.td file which is in the same directory as the RISCVInstrFormats.td file. After inclusion, the “RVInstR” class gets inherited by the “ALU\_rr” class. The “ALU\_rr” class adds the commutability feature which means swapping source 1 and source 2 does not create a different result like in addition but not in subtraction. In the end, XOR’s record is defined by putting funct7, funct3 and assembly string manually in a single line with scheduling information added. 

\section{Adding a New Instruction Using TableGen}\label{sec:MLA_add_section}
We created a tutorial to use as a reference while adding more complex instructions.
Create a new tablegen file for custom additions and include it at the end of the RISCVInstrInfo.td file. We named it RISCVInstrInfoCrypt.td as it is going to be cryptography related.
\begin{lstlisting}[caption= Include file]
include "RISCVInstrInfoCrypt.td"
\end{lstlisting}

Add the specifications of the instruction to the RISCVInstrInfoCrypt.td file. Here we created a new class of instruction is defined named ALU\_rrr. For the MLA instruction, the specifications are:

\begin{lstlisting}
let hasSideEffects = 0, mayLoad = 0, mayStore = 0 in
class ALU\_rrr<bits<2> funct2, bits<3> funct3, string opcodestr,
            bit Commutable = 0>
   : RVInstR4<funct2, funct3, OPC_OP, 
   (outs GPR:$rd), (ins GPR:$rs1, GPR:$rs2, GPR:$rs3),
             opcodestr, "$rd, $rs1, $rs2 ,$rs3"> {
 let isCommutable = Commutable;
}
\end{lstlisting}
Explanation of ALU\_rrr:
\begin{lstlisting}
let hasSideEffects = 0, mayLoad = 0, mayStore = 0 in
\end{lstlisting}

If the instruction has no side effect, hasSideEffects will be 0
If there is no need or possibility to load data from memory, mayLoad will be 0
If there is no need or possibility to store data from memory, mayStore will be 0


\begin{lstlisting}
class ALU_rrr<bits<2> funct2, bits<3> funct3, string opcodestr,
            bit Commutable = 0>
\end{lstlisting}

Class is defined with ALU\_rrr name. Variables are defined. funct2 is two bits of binary number as RVInstR4 is used which reserves 5 bits of funct7 for another register. funct3 is three bits of binary number. opcodestr is the string that will be shown in the assembly file. Commutable is one bit of zero which determines the importance of the order of the inputs.


\begin{lstlisting}
: RVInstR4<funct2, funct3, OPC_OP, (outs GPR:$rd), (ins GPR:$rs1, GPR:$rs2, GPR:$rs3),
\end{lstlisting}

RVInstR4 instruction type is called from RISCVInstrFormats.td file. funct2, funct3, opcode, output, and inputs are entered in function orderly.

\begin{lstlisting}
opcodestr, "$rd, $rs1, $rs2 ,$rs3"> {
 let isCommutable = Commutable;
}
\end{lstlisting}

opcode string and activating commutability option.

Create another file in which we are going to add the schedules. In our case, the name of this file is “RISCVInstrInfoCrypt.td”. The definition of the instruction should be added to this file by using ALU\_rrr class defined above and inputting the correct scheduling variables. For the MLA instruction, it is:

\begin{lstlisting}
def MLA : ALU_rrr<0b10, 0b100, "mla">,
Sched<[WriteIMul, ReadIMul, ReadIMul]>;
\end{lstlisting}

MLA instruction is defined and ALU\_rrr instruction type is used. funct2,funct3, and opcode strings are entered. Schedules are determined.

Add the instruction’s pattern defining its logic to the RISCVInstrInfoCrypt.td file. For the MLA instruction, it is:

\begin{lstlisting}
def : Pat< (add (mul GPR:$src1, GPR:$src2), GPR:$src3),
(MLA GPR:$src1, GPR:$src2, GPR:$src3)>;
\end{lstlisting}

MLA instruction is called from RISCVInstrInfoCrypt.td file and inputs and outputs are determined.



\section{Creating Assembly File From C File}

LLVM consists of many flexible libraries which allows user to use different libraries with preffered  options. To create RISC-V assembly from c code, Clang and LLC are used with the following commands. \\

Clang is the C compiler front-end which is mainly used with LLVM backend. Clang is used in this project to produce LLVM intermediate representation code. Following command produces a .ll file in the current directory. 

\begin{lstlisting}
\$-clang -S -target riscv32-linux-gnu -emit-llvm -g foo.c
\end{lstlisting}

-S option provides to run only preprocess and compilation steps. \\
-target option specifies the 32 bit RISC-V target architecture. \\
-emit-llvm is for targeting the LLVM backend. \\
-g option generates the source level debug information.\\

LLC is the LLVM compiler back-end which converts LLVM intermediate representation (IR) into native machine code for a specific target architecture. Following command produces a .s file for RISC-V architecture in the current directory. 

\begin{lstlisting}
\$ llvm-project/build/bin/llc -debug-only=isel -view-isel-dags -mtriple=riscv32 lxr.ll
\end{lstlisting}

-debug-only=isel option gives the debug informations during the DAG lowering process.\\
-view-isel-dags option prints the directed acylic graph (DAG) image of the IR code.\\
-mtriple=riscv32 defines the 32 bits RISC-V target architecture.\\