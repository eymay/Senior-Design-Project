
\section{C++ ile Desen Eşleştirme}

\subsection{C++ ile Desen Eşleştirme in SelectionDAG}
\begin{frame}[fragile]{TableGen'in Sınırlamaları}
\begin{itemize}
    \item TableGen desenleri yalnızca ağaç yapılandırılmış desenleri eşleştirebilir.
    \item Giriş ilişkileriyle desen eşleştirilemez. Tüm girişler birbirinden bağımsız olmalıdır.
    \item C++ ile karşılaştırıldığında dil tarafından sınırlıdır.
\end{itemize}

\lstinputlisting[caption={Minimal IR Example of xor(load(x), load(x + 16))},linerange={1-5} ,label={lst:sbox-xor},language=llvm,style=nasm]{../s-box/keccakO3.ll}
\end{frame}

\begin{frame}{C++ ile Desen Eşleştirme in SelectionDAG}
\begin{enumerate}
    \item Assembler desteği sağlamak için TableGen'de bir kayıt bildirimi oluşturun, Desen bildirimi şimdilik yoksayılır.
    \item Hata ayıklama çıktısında veya dot dosyasında DAG'ı gözlemleyin ve grafın kökünü bulun. 
    \item Kök talimat durumunda RISCVISelDAGToDAG.cpp dosyasına bir fonksiyon ekleyin. 
    \item SelectionDAG düğümü ile desen eşleştirmeyi ve değiştirmeyi uygulayın.
\end{enumerate}
\end{frame}

\begin{frame}{C++ ile Örnek}{LXR Talimatı}
\begin{itemize}
    \item Örnek talimatımız LXR, bir diziden iki öğe yükler ve bunları XOR yapar.
    \item lxr rd, rs1, rs2 = xor(load(x), load(x + 16))

    \item Talimat bir diziden yüklendiği için, adres değerleri birbirine göreli. 
    \item Bu yüzden TableGen desen eşleştirme için kullanılamaz.
    \item Ancak TableGen her zaman talimat kodlaması ve dolayısıyla Assembler desteği için kullanılabilir.
\end{itemize}
\end{frame}

\begin{frame}[fragile]{TableGen'de LXR'in Talimat Kodlaması}{1.Kayıt Oluştur}
Giriş ve çıkışlar kayıtlar olduğu için TableGen sınıfı R-tipi talimatlar için kullanılabilir.

Özel talimat kodlamamız için funct7 ve funct3'ü geçersiz kılabiliriz.
\begin{lstlisting}
let mayLoad = 1 in{
def LXR : ALU_rr<0b0011011, 0b101, "lxr">,
Sched<[WriteIALU, ReadIALU, ReadIALU]>;
}
\end{lstlisting}
llvm-mc assembler, Assembly dizesinden ikili (binary) yayınlamak için kullanılabilir.
\end{frame}

\begin{frame}[fragile]{LXR Deseninin DAG'ı}{2.DAG'ı Gözlemleyin}
Hedeflenen LXR deseni SelectionDAG'de üretilir ve dökülür (dump()).
\lstinputlisting[caption={Karşılık gelen Optimize Edilmiş ve Yasallaştırılmış DAG}, linerange={3-3,4-4,8-9,13-13,16-16} ,label={lst:sbox-xor-dag},language=llvm,style=nasm]{../s-box/opt-lowered-dag.td}
Talimat Seçimi, SelectionDAG'de optimize edilmiş, yasallaştırılmış DAG üzerinde çalışır.
\end{frame}

\begin{frame}[fragile]{RISCVISelDAGToDAG.cpp'yi Değiştirin}{3.Fonksiyon çağrısı ekleyin}
    RISCVISelDAGToDAG.cpp, C++'da DAG'dan DAG'ye dönüşümleri içerir.
\lstinputlisting[caption={C++'da Desen Eşleştirmesi için Yeni Fonksiyonun Tanıtımı},label={lst:sbox-iseldagtodag},language=C++]{../s-box/custom_c++/iseldagtodagPlace.cpp}
\end{frame}


\begin{frame}[fragile]{C++'da Desen Eşleştirme Fonksiyonu}{4.Desen Eşleştirme Mantığı}
\lstinputlisting[caption={C++ mantığı, Bölüm 1}, label={lst:sbox-cpp}, linerange={1-17} ,language=C++]{../s-box/custom_c++/xor_loads.cpp}
\end{frame}

\begin{frame}[fragile]{C++'da Desen Eşleştirme Fonksiyonu}{4.Desen Eşleştirme Mantığı}
\lstinputlisting[caption={C++ mantığı, Bölüm 2}, label={lst:sbox-cpp2}, linerange={19-33} ,language=C++]{../s-box/custom_c++/xor_loads.cpp}
\end{frame}

\begin{frame}{Fonksiyondaki Desen Kontrolleri}
Kontrol edin eğer..
\begin{enumerate}
    \item .. işlenenler her ikisi de yükleme (load) talimatlarıysa.
    \item .. İkinci yükleme talimatının ikinci operandı bir ekleme (add) talimatıysa.
    \item .. ilk yüklemenin taban ofseti ve ikinci yüklemenin ilk ekleneceği aynıysa, çünkü yapı'nın başlangıcını göstermelidirler.
    \item .. ikinci eklenen sabit bir değerse.
    \item .. sabit ikinci eklenecek değerin işaretsiz değeri 16'ya eşitse.
\end{enumerate}
\end{frame}

\ifAdvanced
    \begin{frame}{SelectionDAG ile İlgili Sorunlar}
    \begin{itemize}
        \item 2005'lerden beri geliştiriliyor
        \item Monolitik, geçişleri (pass) test etmesi zor
        \item Temel bloklar üzerinde çalışır, aynı anda iki fonksiyonun DAG'ına erişemez.
        \item Derleme süresini önemli ölçüde etkiler
        \item Dokümantasyon az
        \item DAG bellekte inşa edildiği için hata ayıklaması zor
    \end{itemize}
    Bu sorunları ele alan yeni alternatif GlobalISel'dir, ancak RISC-V hedefi için hala geliştirme aşamasındadır.

    GlobalISel, AArch64 hedefi için varsayılandır.
\end{frame}

\fi

\subsection{Farklı Seviyelerde C++ ile Desen Eşleştirme}


\begin{frame}{Farklı Seviyelerde Desen Eşleştirme}
Desen eşleştirmeyi arka uçtan (backend) farklı bir seviyede gerçekleştirmek mümkündür.

LLVM IR seviyesi ve MCInst, desen eşleştirmesi için iki alternatiftir.
\begin{center}
    \begin{tikzpicture}[node distance = 2cm, auto]
\tikzstyle{every node}=[font=\footnotesize]
    % Düğümleri yerleştir
    \node [block] (init) {LLVM IR};
    \node [cloud, right of=init,node distance=2.5cm] (ast) {Selection DAG};
    \node [block, right of=ast,node distance=2.5cm] (llvmir) {MCInst};
    % Kenarları çiz
    \path [line] (init) -- (ast);
    \path [line] (ast) -- (llvmir);

   
\end{tikzpicture}
\end{center}
\end{frame}

\begin{frame}{LLVM IR ile Desen Eşleştirme}
Döndürme işlemi "ROTR", IR seviyesinde LLVM IR'de eşleştirilir.
    \begin{itemize}
        \item LLVM IR'de desen eşleştirme, bir içsel fonksiyonunun (intrinsic function) tanımını gerektirir.
        \item İçsel fonksiyon, desene göre optimizasyon geçişlerinde eşleştirilir.
        \item Eşleştirilen özdeşimler arka uca aktarılır ve yeni talimat yayınlanır.
    \end{itemize}
    LLVM IR'de desen eşleştirme daha esnek olabilir, ancak uygulamak daha karmaşık olabilir.
\end{frame}

\begin{frame}{MCInst ile Desen Eşleştirme}
SelectionDAG'den daha düşük bir seviye olan MCInst, talimatlar hakkında daha fazla ayrıntı gerektiğinde kullanılabilir.

"SH1ADD", materyalize edilmiş sabitlerin eşleştirilmesiyle LLVM'de uygulanmıştır. Sabitlerin farklı temsilleri olabilir ve belirli bir sabite ihtiyaç duyulursa, MCInst o sabiti eşleştirmek için daha fazla kontrol sağlar.
\end{frame}
