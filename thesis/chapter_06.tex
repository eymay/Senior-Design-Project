%%%%%%%%%%%%%%%%%%%%%%%%%%%%%%%%%%%%%%%%%%%%%%%%%%%%%%%%%%%%%%%%%%
%% !TEX root = interimreport.tex
%\clearpage
\chapter{REALISTIC CONSTRAINTS AND CONCLUSIONS}\label{Ch6}
%%%%%%%%%%%%%%%%%%%%%%%%%%%%%%%%%%%%%%%%%%%%%%%%%%%%%%%%%%%%%%%%%%

In this project, we aimed to create a compiler backend extension for a custom processor to make the processor compatible with the high-level languages. LLVM compiler infrastructure is used as the compiler design environment. An IBEX core extension is taken into consideration while selecting the instructions that will be matched in the compiler. IBEX ASCON extension core is a RISC-V extension that specifically designed to perform ASCON cryptography algorithm more efficiently and quickly. Thus, the common patterns of the ASCON algorithm are covered and added to our compiler. A new tablegen file is created for our extension instructions. With this extension we managed to match the basic patterns of the ASCON algorithms. We reduced the assembly instruction count that will be generated while compiling a high-level language. We are considering that a target specific compiler will make easy to use this IBEX extension for more complex applications.

In this thesis, common structure of the compilers is presented to give general knowledge about the compilers and expand the reader’s view on compiler structure. Following this, LLVM compiler infrastructure is explained to show the LLVM’s usage and features for the future works in the related area. Then RISC-V and ASCON algorithm are presented briefly to introduce our target hardware’s specifications. To describe the LLVM’s work flow, we followed the path of generating assembly instructions out of high level language codes. We explained how assembly codes are generated through the LLVM’s blocks. After that we showed how to add an instruction to an LLVM backend with an example. Our matched instructions for the ASCON algorithm patterns are described. We also explained how to test a new added instruction’s validities. While we are writing this thesis, we tried to clarify all the works and researces we have done. We wrote this thesis to pioneer the future works in the related area.
\section{Practical Application of This Project}
This project can be useful for increasing the efficiency of applications that require the frequent use of specific instructions. Cryptography applications with RISC-V may be one of these.

\section{Realistic Constraints}
LLVM is a huge infrastructure and while working with it, sometimes it may be hard to find what you are looking for. Also, there aren’t many sources or documentation to find solutions to the specific problems that we encounter which sometimes slows down the progress.

\subsection{Social, Environmental, and Economic Impact}
The end product is going to help the custom processor to be programmed by a high level programming language. It will make the programming of the custom processor a more efficient process and encourage the use of the custom processor. Because of this efficiency, the energy and time costs would be reduced during the programming of the processor. Also, using a custom processor for handling a problem is faster and requires less power. Therefore, encouraging the use of one would be another benefit of the end product.

\subsection{Cost Analysis}
Open-source tools and programs were used on our computers during the project. Therefore, it wasn’t costly for us.

\subsection{Standards}
LLVM project is very selective on the technologies they use. Latest versions of C++ and build tools with software engineering principles are followed. Instructions abide by the RISC-V instruction set standard.

\subsection{Health and Safety Concerns}
Since we are working on software area, there is no possible risk of harm to users.

\section{Future Work and Recommendations}
%bknz form 1/2
