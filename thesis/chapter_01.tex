%%%%%%%%%%%%%%%%%%%%%%%%%%%%%%%%%%%%%%%%%%%%%%%%%%%%%%%%%%%%%%%%%
% !TEX root = interimreport.tex
\chapter{INTRODUCTION}\label{Ch1}
%%%%%%%%%%%%%%%%%%%%%%%%%%%%%%%%%%%%%%%%%%%%%%%%%%%%%%%%%%%%%%%%%
%%TODO Better Intro
%Key points:
%%ASIP are more efficient, fast 
%%Compilers are needed to be modified to run software already developed
%%Without compiler support ASIP becomes limited for acceleration

% give hints about the upcoming chapters 
%%compilers, llvm, riscv, ascon, llvm backend, custom instr, pattern matching and maybe testing
Recent advances and studies on the integrated circuits caused technology to produce application specific circuits for the various areas of usage. Extensions for the open source processor architectures became a part of the industrial development. Especially with RISC-V open and modular ISA, more custom accelerators are developed. Hardware accelerators have the promise of being fast and efficient.

However, loading new abilities to an extended processor comes with a problem. Programming languages and their compilers are developed for common architectures. A compiler targeting standard instruction set architecture (ISA) will not produce the custom instructions needed for the accelerator. A compiler modification is needed to be able to introduce the accelerator to the high level languages. In this thesis, we show various ways to approach the problem and documented best practices for it. 

%TODO Talk about LLVM

%TODO mention ascon for sbox 
For the research, several accelerators with specific custom instructions are targeted \cite{Sairoglu, eryilmaz}. Instructions which are targeted to hardware are SHLXOR, RORI and S-box. The encodings and instruction operations were mostly designed by the hardware developers. The process of required compiler modifications for SHLXOR and RORI are demonstrated in Sections \ref{sec:shlxor} and \ref{sec:rori}. S-box instruction, due to its non-linearity, was a complicated instruction to characterize. It is a good example that not every instruction can be added in a similar process and instruction specific design can be required. 
%TODO talk about instruction design and riscv, maybe in separate paragraph after the one below
Also similar to the design of ISA's, instructions should be designed by considering both hardware and software. 

S-box instruction is analyzed from several aspects. Firstly, the IR optimizations it gets through are demonstrated in depth in Section \ref{sbox-case}. Secondly, the limitations of TableGen which was a sufficient system for the previous instructions, are discussed and C++ pattern matching is explained in Section \ref{sec:cpp}. 
Thirdly, pattern matching in IR and MCInst level are discussed in Section \ref{sec:patmatchdisc}. Finally, we proposed two new instructions that can be implemented in hardware that can accelerate S-box operation as LXR and NAXOR. A simplified version of LXR which has independant Load addresses is demonstrated in Section \ref{sec:lxr}. In S-box case where the load addresses are dependant is presented in C++ pattern matching in Section \ref{sec:cpp}. The second proposed instruction, NAXOR, is presented in Section \ref{sec:naxor}.

In conjunction with LXR and NAXOR which do not have a target hardware, MLA instruction is also presented without target hardware. MLA is discussed in detail in Chapter \ref{Ch4} where it is traced from the C code to Assembly in steps of compilation and Section \ref{sec:MLA_add_section} where its support was added with TableGen. 

%TODO Pattern Matching 

\section{Purpose of Project}
Application-Specific Instruction Set Processors (ASIP) are becoming more popular with the development of embedded systems. The specialization of the core causes a tradeoff between flexibility and performance. For special purposes, using ASIPs increases efficiency however, we can program a custom ASIP only by using assembly instructions that we defined. Programming custom processors with assembly language is not a preferred way of coding. We are also not able to use high-level languages because compiling tools are designed for common architectures with certain instructions. The ability to add custom instructions to compilers will enable us to make more use of custom hardware designs.

ASIPs are feasible for all application-specific embedded systems like consumer, industrial, automotive, home appliances, cryptology, medical, telecommunication, commercial, aerospace, and military applications. The custom back-end that we will design under the supervision of Dr. Tankut Akgül, is going to serve the processor designed by Prof. Dr. Sıddıka Berna Örs Yalçın’s research team. When the project is completed, Prof. Yalçın is going to be able to produce the assembly codes that are compatible with the processor’s extended instruction set in addition to RISC-V.

There are two ways to avoid designing a specific compiler for embedded microprocessors. The first one is using a common processor that already has compiling tools. The advantage of it is reducing the costs for both hardware and software designs. However, it reduces efficiency dramatically because the processor is not designed for a specific task, and hardware cost increases while the speed is decreasing. Another way is using an application-specific instruction set processor and programming with the assembly instructions that the hardware designer defined. Theoretically, all efficiency benefits can be achieved but programming with assembly instructions increases coding difficulty excessively.

Prof. Yalçın and her team are studying designing application-specific instruction set processors. The purpose of this project is to create a compiler back-end for a processor that supports custom instructions on top of RISC-V instructions. This compiler is going to help to program the custom processor by using high-level languages. Existing RISC-V compilers are not able to produce efficient assembly codes for ASIPs. Therefore a need arose for a compiler back-end.
The main reason for choosing this project is that we wanted to meet an actual need for a critical existing problem. The project has the potential to be the bridge between hardware and software of custom hardware projects in research, enabling them to be candidates for critical applications. Our team has skills and experience in low-level programming and digital design. Our project advisor Dr. Tankut Akgül’s lecture on microprocessors led us to work with him on the embedded systems. Our team member Mehmet Eymen Ünay is a double major student in the computer engineering department. Bora İnan has experience in software development in the defense industry and Emrecan Yiğit is working on low-level robotics programming. Emrecan and Bora are taking a digital system Design and application course from Prof. Yalçın to learn Verilog and get to know about hardware design. All of us have assembly, C, and FPGA programming experience. We also have worked with several open-source frameworks on different Linux distros. We consider that our skills and experiences match the project we will be doing.

