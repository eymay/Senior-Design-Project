Özel komutlara sahip donanım hızlandırıcılarının yükselişi, bu hızlandırıcıları destekleyen özel derleyici arka uçlarını gerektirmektedir. Bu çalışma, LLVM ve RISC-V arka ucunun ayrıntılı analizlerini sunmakta ve söz konusu dönüşümlere uçtan uca genel bakış sağlayan vaka çalışmalarıyla desteklenmektedir. 

Komut tasarımının hem donanım hem de yazılım tasarım alanını dikkate alması gerektiğini tartışıyoruz. Gerekli derleyici değişiklikleri, komutun iyi tasarlanmadığı ve yeniden gözden geçirilmesi gerektiği anlamına gelebilir. RISC-V standart uzantılarının komut tasarımcılarına rehberlik edebilecek örnek komutlar sağladığını tartışıyoruz.

Bu çalışmada derleyiciye özel bir komut ekleme süreci Assembler desteği ve desen eşleştirme desteği olarak iki kısma ayrılmıştır. Desen eşleştirme desteği olmadan, geleneksel yazılımlar hızlandırıcı için manuel satır içi Assembly girişleri gerektirir ve bu da ölçeklenebilir değildir. Komut semantiğinden bağımsız olarak Assembler desteği eklemek önemsiz olsa da, desen eşleştirme desteğinde durum tam tersidir. Desen eşleştirme desteği ve değişiklik için doğru aşamayı seçmek, derleyicideki iç dönüşümlerin bilinmesini gerektirir. Bu çalışma örüntü eşleştirme konusunu derinlemesine incelemekte ve örüntü eşleştirme desteği sorununa yaklaşmanın çeşitli yollarını sunmaktadır. Örüntünün karmaşıklığına bağlı olarak, daha yüksek seviyeli dönüşümlerin, örneğin IR seviyesinin, Komut Seçimi aşamasına kıyasla daha sürdürülebilir olabileceği tartışılmaktadır.

%Özet hazırlanırken 1 satır boşluk bırakılır. Türkçe tezlerde, Türkçe özet 300 kelimeden
%az olmamak kaydıyla 1-3 sayfa, İngilizce genişletilmiş özet de 3-5 sayfa arasında olmalıdır.

%İngilizce tezlerde ise, İngilizce özet 300 kelimeden az olmamak kaydıyla 1-3 sayfa, Türkçe genişletilmiş özet de 3-5 sayfa arasında olmalıdır.

%Özetlerde tezde ele alınan konu kısaca tanıtılarak, kullanılan yöntemler ve ulaşılan sonuçlar belirtilir. Özetlerde kaynak, şekil, çizelge verilmez.

%Özetlerin başında, birinci dereceden başlık formatında tezin adı (önce 72, sonra 18  punto  aralık  bırakılarak  ve  1  satır  aralıklı  olarak)  yazılacaktır.   Başlığın  altına büyük harflerle sayfa ortalanarak (Türkçe özet için) \textbf{ÖZET} ve (İngilizce özet için) SUMMARY yazılmalıdır.

%Türkçe tezlerde Türkçe özetin İngilizce özetten önce olması önerilir.

%1 line spacing must be set for summaries. For theses in Turkish, the summary in Turkish must have 300 words minimum and span 1 to 3 pages, whereas the extended summary in English must span 3-5 pages.
%For theses in English, the summary in English must have 300 words minimum and span 1-3 pages, whereas the extended summary in Turkish must span 3-5 pages.
%A summary must briefly mention the subject of the thesis, the method(s) used and the conclusions derived.
%References, figures and tables must not be given in Summary.
%Above the Summary, the thesis title in first level title format (i.e., 72 pt before and 18 pt after paragraph spacing, and 1 line spacing) must be placed. Below the title, the expression \textbf{ÖZET} (for summary in Turkish) and \textbf{SUMMARY} (for summary in English) must be written horizontally centered.
%It is recommended that the summary in English is placed before the summary in Turkish.

