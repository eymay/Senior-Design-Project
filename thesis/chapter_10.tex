%%%%%%%%%%%%%%%%%%%%%%%%%%%%%%%%%%%%%%%%%%%%%%%%%%%%%%%%%%%%%%%%%
% !TEX root = interimreport.tex
\clearpage
\chapter{TESTING}\label{Ch10}
%%%%%%%%%%%%%%%%%%%%%%%%%%%%%%%%%%%%%%%%%%%%%%%%%%%%%%%%%%%%%%%%%
Regression testing is a core part of LLVM because of its size and active development. To make sure newly added features don’t break the already present functionality it is a must to both build functionality and its corresponding tests.

LLVM-lit coordinates the testing procedure. The comment lines that start with “RUN” call other programs via LLVM-lit. LLVM-lit also gives the output of a program to another program as an input.

FileCheck, as the name implies, controls the checking process. It basically compares the file and the corresponding lines of the output. It is used with CHECK-* command.

In this chapter SHLXOR instruction which was introduced in Section \ref{sec:shlxor} will be tested. Its instruction encoding and Assembler support are tested with MC tests. Its pattern matching support is tested with LLC tests.   
\section{MC Test}
LLVM-MC is an abstracted assembler and object file emitter integrated with the compiler \cite{Lattner2010Apr}.

Before adding a new instruction, its encoding should be designed. LLVM-MC is going to be used to get the encoding results. A wrong input to TableGen class may result in a different encoding than expected. With an MC test, the expectation and result will be compared. 

\begin{minipage}{\linewidth}
\begin{lstlisting}[caption={MC Test File}, label={lst:mc_test_file} ]
# RUN: llvm-mc %s -triple=riscv32 -riscv-no-aliases -show-encoding \ 
# RUN: 	| FileCheck -check-prefixes=CHECK-ASM,CHECK-ASM-AND-OBJ %s 
# RUN: llvm-mc -filetype=obj -triple=riscv32 < %s \ 
# RUN: 	| llvm-objdump -M no-aliases -d -r - \ 
# RUN: 	| FileCheck -check-prefixes=CHECK-ASM-AND-OBJ %s 

# CHECK-ASM-AND-OBJ: shlxor s2, s2, s8 
# CHECK-ASM: encoding: [0x33,0x59,0x89,0x81] 
shlxor s2, s2, s8
\end{lstlisting}
\end{minipage}


The first RUN command sequence results in the output given in Code \ref{lst:result_of_first_run_command}.

\begin{minipage}{\linewidth}
\begin{lstlisting}[language=sh, caption={Result of first "RUN" command}, label={lst:result_of_first_run_command} ]
$ <llvm-build-path>/build/bin/llvm-mc \\
<llvm-build-path>/llvm/test/MC/RISCV/crypt.s \\ 
-triple=riscv32 -riscv-no-aliases -show-encoding

             . text

             shlxor s2, s2, s8 # encoding: [0x33,0x59, 0x89, 0x81]
\end{lstlisting}
\end{minipage}

In Code \ref{lst:mc_test_file}, FileCheck is checking both the Assembly encoding and the string as it was provided with the prefixes: CHECK-ASM, CHECK-ASM-AND-OBJ.


The second RUN sequence with only the llvm-mc part gives out an ELF object which itself isn’t useful. The console output can be seen in Code \ref{lst:result_of_second_run_command}.


\begin{lstlisting}[language=sh, caption={Result of second "RUN" command}, label={lst:result_of_second_run_command} ]
$ <llvm-build-path>/build/bin/llvm-mc \\
<llvm-build-path>/llvm/test/MC/RISCV/crypt.s \\ 
-filetype=obj -triple=riscv32 

ELF`4(3Y.text.strtab.symtab48%
\end{lstlisting}
By using the pipe ‘|’ operator similar to Shell usage, we can tell LLVM-lit to feed another program with the output of a program. The test commands with shell pipe are demonstrated in Code \ref{lst:use_of_pipe_operator}.


\begin{lstlisting}[language=sh, caption={Use of pipe operator in Shell}, label={lst:use_of_pipe_operator} ]
$ <llvm-build-path>/build/bin/llvm-mc -filetype=obj -triple=riscv32 <
<llvm-build-path>/llvm/test/MC/RISCV/crypt.s | \\
<llvm-build-path>/build/bin/llvm-objdump -M no-aliases -d -r -


<stdin>: file format elf32-littleriscv

Disassembly of section .text:

00000000 <.text>:
       0: 33 59 89 81 = shlxor s2, s2, s8
\end{lstlisting}

Observing the outputs directly through the shell indicates what FileCheck is looking for in the standard output. Here in the second RUN sequence, FileCheck is provided only with CHECK-ASM-AND-OBJ and therefore it does not check the CHECK-ASM line. It seems that the object dump resulted in the correct string.

In Code \ref{lst:output_for_successfully_passed_test} we can observe that the test is passed:

\begin{minipage}{\linewidth}
\begin{lstlisting}[language=sh, caption={Output for successfully passed test}, label={lst:output_for_successfully_passed_test} ]
$ <llvm-build-path>/build/bin/llvm-lit \\
<llvm-build-path>/llvm/test/MC/RISCV/crypt.s -v

-- Testing: 1 tests, 1 workers --
PASS: LLVM :: MC/RISCV/crypt.s (1 of 1)

Testing Time: 0.07s
Passed: 1
\end{lstlisting}
\end{minipage}

Another complexity LLVM-lit handles is the path of our compiled binaries. The test case is free of the paths and \%s placeholders are populated by LLVM-lit.



\section{LLC/CodeGen Test}\label{sec:llc_test}
LLC tests are more familiar since they check how an LLVM IR file produces Assembly strings. Instead of manually checking whether an instruction is emitted in the output, this tool can be used and batch testing can be done.

An example is given in Code \ref{lst:minimal_func} of the minimal LLVM IR code implemented producing the SHLXOR instruction we want. As can be seen from its signature, it is a function taking two 32-bit integers and returning one. It takes one input, shifts it left by one and assigns it to a variable \%1. \%1 is then XOR’ed with the second input and the result is returned.

\begin{lstlisting}[language=llvm,style=nasm, caption={Minimal shlxor producing LLVM IR}, label={lst:minimal_func} ]
; RUN: llc -mtriple=riscv32 -verify-machineinstrs < %s \ 
; RUN: | FileCheck %s -check-prefix=RV32R 


define i32 @shlxor(i32 %a, i32 %b) nounwind { 
%1 = shl i32 %a, 1 
%2 = xor i32 %1, %b 
ret i32 %2 
} 
\end{lstlisting}




A utility script as its usage is shown in Code \ref{lst:command_for_using_utility_script}, can be used to generate the expected result and insert it to the test file.

\begin{lstlisting}[language=sh, caption={Command for using utility script}, label={lst:command_for_using_utility_script} ]
$ <llvm-build-path>/utils/update_llc_test_checks.py \\
--llc-binary <llvm-build-path>/build/bin/llc shlxor.ll
\end{lstlisting}

The LLC test file is populated with FileCheck lines as seen in Code \ref{lst:llc_test_file}.

\begin{minipage}{\linewidth}
\begin{lstlisting}[language=llvm,style=nasm, caption={The final LLC test file}, label={lst:llc_test_file} ]
NOTE: Assertions have been autogenerated by utils/update_llc_test_checks.py
; RUN: llc -mtriple=riscv32 -verify-machineinstrs < %s \ 
; RUN: | FileCheck %s -check-prefix=RV32R 


define i32 @shlxor(i32 %a, i32 %b) nounwind { 
; RV32R-LABEL: shlxor:
; RV32R:       # %bb.0:
; RV32R-NEXT:    shlxor    a0, a0, a1
; RV32R-NEXT:    ret
%1 = shl i32 %a, 1 
%2 = xor i32 %1, %b 
ret i32 %2 
} 
\end{lstlisting}
\end{minipage}


We can observe that our newly added SHLXOR instruction is recognized and placed in the check lines. When LLVM-lit is run with the test file, the test is passed as the new instruction is implemented. Output for a successful test is given in Code \ref{lst:output_for_passed_ll_test}.

\begin{lstlisting}[language=sh, caption={Output for successfully passed test}, label={lst:output_for_passed_ll_test} ]
$ <llvm-build-path>/build/bin/llvm-lit shlxor.ll -v


-- Testing: 1 tests, 1 workers --
PASS: LLVM :: CodeGen/RISCV/shlxor.ll (1 of 1)

Testing Time: 0.08s
Passed: 1
\end{lstlisting}

This way a large number of instructions can be tested and the effect of our modifications can be analyzed by running the tests.
