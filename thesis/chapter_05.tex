%%%%%%%%%%%%%%%%%%%%%%%%%%%%%%%%%%%%%%%%%%%%%%%%%%%%%%%%%%%%%%%%%
% !TEX root = interimreport.tex
\clearpage
\chapter{ADDING CUSTOM INSTRUCTIONS}\label{ch:custom_instr}
%%%%%%%%%%%%%%%%%%%%%%%%%%%%%%%%%%%%%%%%%%%%%%%%%%%%%%%%%%%%%%%%%
The instruction selection system we focused on at the back end of the LLVM compiler is SelectionDAG among FastISel and GlobalIsel. SelectionDAG is the most mature Instruction Selection framework with more target support. However, shortly it is worth considering GlobalISel as it is developed recently as an alternative to SelectionDAG. The reasons to replace it are to make it faster, smaller, more testable and open to low-level optimizations.
\section{TableGen Reference}
TableGen is a domain-specific language used in the LLVM back end side to generate CPP header files. The purpose it serves is that it removes the redundancy of instruction declaration code which can be common to numerous architectures with minor differences. To maintain and scale the framework the minor differences are implemented level by level at a series of inheritance operations between TableGen classes. 

\par
 LLVM Static Compiler, LLC, is responsible for converting LLVM IR to Assembly codes. To add new instructions, changes are made in TableGen files and LLC is recompiled. During the compilation operation of the LLC program, TableGen records are created which declare every instruction’s encoding and describe its features. Referring to the records, DAGs are used in the process of instruction selection. 
 DAG is a graph structure that has no cycles and has directions on the edges.

 Operations or functions are represented as nodes in the DAG. They are critical parts of declaring the logic or pattern of the new instruction. 
\par

The operations represented on the DAG can be LLVM intrinsics as well as instructions. LLVM instructions resemble conventional assembly instructions, in contrast, LLVM intrinsics have higher level abstraction depending on their functionality. Their instruction generation may vary depending on the target hardware. It is possible to define a new complicated instruction either by combining simple LLVM instructions and higher-level intrinsics in the DAG level or by creating a new LLVM intrinsic which gets created at the Intermediate Level of the compilation process.

\section{RISC-V TableGen Classes}
The most general instruction class used for every target architecture is the “InstructionEncoding” TableGen class defined in llvm/include/llvm/Target/Target.td. This class holds the decoder method and size of instruction in addition to minor variables. It gets inherited by the generic “Instruction” class which is defined in the same class. This class holds input and output DAGs and information which is useful to the compiler and is generalizable to all architectures.
\par

The general class gets inherited by every target-specific class. In RISC-V’s case, the next stop of the instruction is the “RVInst” class which inherits from the general “Instruction” class and it resides in llvm/lib/Target/RISCV/RISCVInstrFormats.td TableGen file. It defines the general bit patterns of RISC-V instructions. For example, the opcode being the first 7 bits. It defines additional information like the assembly string pattern. This general class is inherited by every type of instruction of R, I, S, B, U, and J types. As a simple example, XOR instruction can be traced. As XOR is an R type, a register-register instruction, it continues its inheritance journey from “RVInstR”. It is common to R type instructions to have funct7, rs2, rs1, funct3, and rd format ordered from most significant bit (MSB) to least significant bit (LSB). These variables are assigned corresponding bit fields in the class. 
\par

The RISC-V formats mentioned are included in the llvm/lib/Target/RISCV/RISCVInstrInfo.td file which is in the same directory as the RISCVInstrFormats.td file. After inclusion, the “RVInstR” class gets inherited by the “ALU\_rr” class. The “ALU\_rr” class adds the commutability feature which means swapping source 1 and source 2 does not create a different result like in addition but not in subtraction. In the end, XOR’s record is defined by putting funct7, funct3 and assembly string manually in a single line with scheduling information added. 

\pagebreak
\section{Adding a New Instruction Using TableGen}\label{sec:MLA_add_section}

This section will guide the reader in introducing new instructions via TableGen.
Create a new TableGen file for custom additions and include it at the end of the RISCVInstrInfo.td file. We named it RISCVInstrInfoCrypt.td as it is going to be cryptography related.
\begin{lstlisting}[caption= Include file]
include "RISCVInstrInfoCrypt.td"
\end{lstlisting}

The specifications of the instruction will be added to the RISCVInstrInfoCrypt.td file.

\subsection{Introducing the ALU\_rrr TableGen Class}\label{sec:alurrr}

Here we created a new class of instruction named ALU\_rrr. MLA instruction requires three source registers and is defined to be ALU type so the specifications are:

\begin{lstlisting}[caption={ALU\_rrr class definition},label={lst:ALU_rrr}]
let hasSideEffects = 0, mayLoad = 0, mayStore = 0 in
class ALU_rrr<bits<2> funct2, bits<3> funct3, string opcodestr,
            bit Commutable = 0>
   : RVInstR4<funct2, funct3, OPC_OP, 
   (outs GPR:$rd), (ins GPR:$rs1, GPR:$rs2, GPR:$rs3),
             opcodestr, "$rd, $rs1, $rs2 ,$rs3"> {
 let isCommutable = Commutable;
}
\end{lstlisting}
The class is wrapped with three flags:
\begin{lstlisting}
let hasSideEffects = 0, mayLoad = 0, mayStore = 0 in
\end{lstlisting}

\begin{itemize}
    \item If the instruction has no side effect, \textit{hasSideEffects} will be 0.
    \item If there is no need or possibility to load data from memory, \textit{mayLoad} will be 0.
    \item If there is no need or possibility to store data from memory, \textit{mayStore} will be 0.
\end{itemize}


Here you can see class arguments:
\begin{lstlisting}
class ALU_rrr<bits<2> funct2, bits<3> funct3, string opcodestr,
            bit Commutable = 0>
\end{lstlisting}

Class is defined with ALU\_rrr name. Variables are defined. \textit{funct2} is a two-bit binary number as RVInstR4 is used which reserves 5 bits of \textit{funct7} for another register. \textit{funct3} is a three-bit binary number. \textit{opcodestr} is the string that will be shown in the assembly file. \textit{Commutable} is a zero bit which determines the importance of the order of the inputs.

\begin{lstlisting}
: RVInstR4<funct2, funct3, OPC_OP,
(outs GPR:$rd), (ins GPR:$rs1, GPR:$rs2, GPR:$rs3),
\end{lstlisting}

RVInstR4 instruction type is called from RISCVInstrFormats.td file. \textit{funct2}, \textit{funct3}, \textit{opcode}, output and inputs are given as arguments to the higher class in order.

\begin{lstlisting}
opcodestr, "$rd, $rs1, $rs2 ,$rs3"> {
 let isCommutable = Commutable;
}
\end{lstlisting}

Opcode string used for Assembly and activating commutability option.

\subsection{Introducing the MLA Instruction for Assembler Support}

The definition of the instruction can now be added using the ALU\_rrr class defined above and by choosing the correct scheduling variables. For the MLA instruction, it is:

\begin{lstlisting}
def MLA : ALU_rrr<0b10, 0b100, "mla">,
Sched<[WriteIMul, ReadIMul, ReadIMul]>;
\end{lstlisting}

MLA instruction is defined and ALU\_rrr instruction type is used. funct2,funct3, opcode string and schedules are sufficient to have the full definition of the instruction thanks to the custom ALU\_rrr class.

\subsection{Introducing the MLA Instruction for Pattern Matching Support}
Add the instruction’s pattern defining source to the target custom instruction transformation. For the MLA instruction, it is:

\begin{lstlisting}
def : Pat< (add (mul GPR:$src1, GPR:$src2), GPR:$src3),
(MLA GPR:$src1, GPR:$src2, GPR:$src3)>;
\end{lstlisting}

Note that it is possible to define more patterns to introduce optimizations. There can be multiple source patterns for the same target pattern. The target pattern can also be a tree of SDNodes containing the custom instruction.

\section{Adding Pattern Matching Support for New Instruction Using C++ in SelectionDAG}\label{sec:cpp}
TableGen aims to provide a declarative way to introduce new patterns for new instruction developers. However, not all instructions can be described in this scheme.  Although it is called "dag" as a keyword in TableGen, it expects a tree of instructions. For certain use cases, custom C++ can be the only way to match until the TableGen based system improves. It is also possible to use C++ in complex patterns together with TableGen, which can make the most of the pattern declarative and only the necessary part in imperative style.
\par
A domain that TableGen fails is matching a graph of instructions with dependant operands. As an example, we can think of an instruction having two operands of Load Instructions. If the Load instructions are from an array, they must be related to each other by an offset and it might need to be detected for certain patterns.


\lstinputlisting[caption={Minimal Subtree of Optimized S-box LLVM IR},linerange={1-5} ,label={lst:sbox-xor},language=llvm,style=nasm]{../s-box/keccakO3.ll}

In Code \ref{lst:sbox-xor}, it can be seen that the XOR instruction is between the first element of the input struct and the fifth element of it. As the locations of elements matter, they must be matched by not only looking at Load instructions but also their operands. What we are looking for is to have the base of Load instruction to be equal and the offset operand of it to evaluate to 4, designating the fifth element of the struct. TableGen is not suitable for this operation and the source code of SelectionDAG's RISC-V backend should be analyzed to place the logic to pattern match this set of instructions.

The process for adding an instruction via C++ is as follows:
\begin{enumerate}
    \item Create a record declaration in TableGen to provide the Assembler support, ignoring the Pattern declaration.
    \item Observe the DAG in the debug output or dot file and locate the root of it. 
    \item Add a function in RISCVISelDAGToDAG.cpp file in the root instruction case. 
    \item Implement pattern matching and replacement with SDnode.
\end{enumerate}

SelectionDAG consists of numerous files but the most relevant ones to the developer can be few if the complexity of the pattern is small. The order of files will be from IR to Assembly. RISCVCodeGenPrepare.cpp file provides mechanisms for matching in IR form. This file exists mainly due to the limitation of SelectionDAG which is running per basic block. RISCVISelLowering.cpp contains the lowering of IR to SDnodes. It can decide on whether a type or expression should be legalized or expanded. RISCVISelDAGToDAG.cpp is the instruction selector in C++. Its implementation traverses the DAG from the root and runs the selection functions depending on the SDnode type which is parallel to instructions. 


\lstinputlisting[caption={The corresponding Optimized and Legalized DAG of Code \ref{lst:sbox-xor}},linerange={3-3,4-4,8-9,13-13,16-16} ,label={lst:sbox-xor-dag},language=llvm,style=nasm]{../s-box/opt-lowered-dag.td}

In an attempt to match the three instructions in Code \ref{lst:sbox-xor}, the DAG in Code \ref{lst:sbox-xor-dag} is analyzed. The first remark is that "getelementptr" is converted to an add instruction which calculates the offset. Another remark is that the load instructions have the same base address in the first operand as expected pointing to the same node. If the pattern is large, we can introduce a new function which will contain the logic. The function's prototype should be added to the corresponding header file.


\lstinputlisting[caption={Introduction of New Function for Pattern Matching in C++},label={lst:sbox-iseldagtodag},language=C++]{../s-box/custom_c++/iseldagtodagPlace.cpp}


The C++ logic for matching this pattern is provided below.

\lstinputlisting[caption={C++ logic for Pattern Matching the DAG in Code \ref{lst:sbox-xor-dag}},label={lst:sbox-cpp},language=C++]{../s-box/custom_c++/xor_loads.cpp}

The logic starts from the root of the DAG which in this case is XOR instruction or the ISD::XOR SDnode. Then we iterate through its leaves and check the distinctive features of the pattern. In this pattern, we are interested in the distance of offset addresses of Load instructions so we progressively approach them by assuming the pattern holds and quitting if not. Progressive checking is a common theme in LLVM and as Instruction Selection is one of the stages affecting the compilation times significantly, patterns should not be checked in a single if statement by logical combinations. 

The checks, if statements are doing can be summarised in steps. As the function only runs when the instruction selection finds an XOR SDNode, the root can be assumed as XOR safely.
\begin{enumerate}
    \item Check if both the operands are Load instructions.
    \item Check if the second Load instruction has an Add instruction in its second operand
    \item Check if the base offset of the first load and the first addendum of the second load are the same, as they should point to the beginning of the struct.
    \item Check if the second addendum is a constant.
    \item Check if the unsigned value of constant second addendum is equal to 16.
\end{enumerate}

At this point, it can be safely assumed that the only XOR that conforms to the pattern can be in this line of program. SelectionDAG provides more API to simply replace the nodes in that pattern with the custom instruction.


To interact with the DAG, SelectionDAG's API is used. We encourage the developers to read the source code and learn to use the public functions exposed by SelectionDAG in order to interact with the DAG most effectively. RISCVISelDAGToDAG.cpp file already has many instruction selection mechanisms in place which can be read through. 

\section{Discussion of Pattern Matching in Other Stages of the Compiler}\label{sec:patmatchdisc}
Pattern Matching can be assumed to mainly be an Instruction Selection problem where the pattern will be simply identified and replaced. However, when we take a look at the baseline problem any Compiler technology solves, it is to convert more familiar patterns in some language to a more unfamiliar pattern in machine language. Also, this conversion occurs in a large number of steps through optimization in IR form as discussed in Section \ref{sec:opt}, to DAG formation in SelectionDAG to MCInstr form down the pipeline. Their data structures can differ in representing the instructions which can make some pattern matching schemes to be more fragile than others. Also, as the lowering gets performed high-level information about the program is lost but the formation gets closer to the final output of Assembly. 

For simple cases where for example a combination of R-type instructions will be matched and replaced, Instruction Selection might be the most convenient stage to extend. However if the pattern requires the instruction selection to be performed already, pattern match can be done in the MC layer. On the contrary, if higher level information of the pattern is required, a pattern can be matched to an intrinsic function at the IR level. 

Another reason to consider different stages is that there can be multiple patterns mapped to the same instruction. Further optimization opportunities can rise in further stages. 

\subsection{Case Study: SH1ADD in SelectionDAG and MC Layer}

It was discussed that dealing with the lowered DAG to MC layer can provide more optimization opportunities. In this section, the case of "SH1ADD" instruction which is ratified in the RISC-V Zba extension will be analyzed. The instruction shifts rs1 left by one, adds rs2 and writes to rd. Its encoding and pattern in the standard implementation of LLVM in TableGen are as follows:

\begin{lstlisting}[%language=TableGen,
caption={Instruction Encoding of the Instructions} ]
let Predicates = [HasStdExtZba] in {
def SH1ADD : ALU_rr<0b0010000, 0b010, "sh1add">,
             Sched<[WriteSHXADD, ReadSHXADD, ReadSHXADD]>;
def SH2ADD : ALU_rr<0b0010000, 0b100, "sh2add">,
             Sched<[WriteSHXADD, ReadSHXADD, ReadSHXADD]>;
def SH3ADD : ALU_rr<0b0010000, 0b110, "sh3add">,
             Sched<[WriteSHXADD, ReadSHXADD, ReadSHXADD]>;
} // Predicates = [HasStdExtZba]
\end{lstlisting}

As the SH2ADD and SH3ADD have similar implementations to SH1ADD their patterns will be stripped.

\begin{lstlisting}[%language=TableGen
, caption={Instruction Pattern of the Instructions}]
let Predicates = [HasStdExtZba] in {
def : Pat<(add (shl GPR:$rs1, (XLenVT 1)), non_imm12:$rs2),
          (SH1ADD GPR:$rs1, GPR:$rs2

// More complex cases use a ComplexPattern.
def : Pat<(add sh1add_op:$rs1, non_imm12:$rs2),
          (SH1ADD sh1add_op:$rs1, GPR:$rs2)>;)>;

def : Pat<(add (mul_oneuse GPR:$rs1, (XLenVT 6)), GPR:$rs2),
          (SH1ADD (SH1ADD GPR:$rs1, GPR:$rs1), GPR:$rs2)>;
def : Pat<(add (mul_oneuse GPR:$rs1, (XLenVT 10)), GPR:$rs2),
          (SH1ADD (SH2ADD GPR:$rs1, GPR:$rs1), GPR:$rs2)>;
def : Pat<(add (mul_oneuse GPR:$rs1, (XLenVT 18)), GPR:$rs2),
          (SH1ADD (SH3ADD GPR:$rs1, GPR:$rs1), GPR:$rs2)>;
\end{lstlisting}

We can observe that ComplexPattern's  are used to enable using C++ together with TableGen. The Complex Pattern's TableGen declarations are provided below:

\begin{minipage}{\linewidth}
\begin{lstlisting}[language=C++, caption={TableGen Declaration of ComplexPatterns}]

def sh1add_op : ComplexPattern<XLenVT, 1, 
                        "selectSHXADDOp<1>", [], [], 6>;

class binop_oneuse<SDPatternOperator operator>
    : PatFrag<(ops node:$A, node:$B),
              (operator node:$A, node:$B), [{
  return N->hasOneUse();
}]>;

def mul_oneuse : binop_oneuse<mul>;
\end{lstlisting}
\end{minipage}

"selectSHXADDOp" is a template function which provides the shift amount argument.

\begin{lstlisting}[language=C++, caption={Template Function of the ComplexPattern for "sh1add\_op"}]

bool selectSHXADDOp(SDValue N, unsigned ShAmt, SDValue &Val);
template <unsigned ShAmt> bool 
                      selectSHXADDOp(SDValue N, SDValue &Val) {
return selectSHXADDOp(N, ShAmt, Val);
}
\end{lstlisting}

The C++ logic can be found in RISCVISelDAGToDAG.cpp file, the implementation will be reduced to the patterns described in the comments:

\begin{lstlisting}[language=C++, caption={Implementation of the ComplexPattern for "sh1add\_op"}]
/// Look for various patterns that can be done with a SHL that can be 
/// folded into a SHXADD. \p ShAmt contains 1, 2, or 3 and is set based 
/// on which SHXADD we are trying to match.
bool RISCVDAGToDAGISel::selectSHXADDOp(SDValue N, unsigned ShAmt,
                                       SDValue &Val) {
  if (N.getOpcode() == ISD::AND 
                      && isa<ConstantSDNode>(N.getOperand(1))) {
    SDValue N0 = N.getOperand(0);

    bool LeftShift = N0.getOpcode() == ISD::SHL;
    if ((LeftShift || N0.getOpcode() == ISD::SRL) &&
        isa<ConstantSDNode>(N0.getOperand(1))) {
      uint64_t Mask = N.getConstantOperandVal(1);
      unsigned C2 = N0.getConstantOperandVal(1);

      unsigned XLen = Subtarget->getXLen();
      if (LeftShift)
        Mask &= maskTrailingZeros<uint64_t>(C2);
      else
        Mask &= maskTrailingOnes<uint64_t>(XLen - C2);

      // Look for (and (shl y, c2), c1) where c1 is a shifted mask 
      // with no leading zeros and c3 trailing zeros. We can use an 
      // SRLI by c2+c3 followed by a SHXADD with c3 for the X amount.
       ...
        // Look for (and (shr y, c2), c1) where c1 is a shifted mask 
        // with c2 leading zeros and c3 trailing zeros. We can use an
        // SRLI by C3 followed by a SHXADD using c3 for the X amount.
        ...
      }
    }
  }

  bool LeftShift = N.getOpcode() == ISD::SHL;
  if ((LeftShift || N.getOpcode() == ISD::SRL) &&
      isa<ConstantSDNode>(N.getOperand(1))) {
    SDValue N0 = N.getOperand(0);
    if (N0.getOpcode() == ISD::AND && N0.hasOneUse() &&
        isa<ConstantSDNode>(N0.getOperand(1))) {
      uint64_t Mask = N0.getConstantOperandVal(1);
      if (isShiftedMask_64(Mask)) {
        unsigned C1 = N.getConstantOperandVal(1);
        unsigned XLen = Subtarget->getXLen();
        unsigned Leading = XLen - llvm::bit_width(Mask);
        unsigned Trailing = llvm::countr_zero(Mask);
        // Look for (shl (and X, Mask), C1) where Mask has 32 leading 
        // zeros and C3 trailing zeros. 
        // If C1+C3==ShAmt we can use SRLIW+SHXADD.
        ...
        // Look for (srl (and X, Mask), C1) where Mask has 32 leading
        // zeros and C3 trailing zeros. 
        // If C3-C1==ShAmt we can use SRLIW+SHXADD.
        ...
      }
    }
  }
  return false;
}
\end{lstlisting}

Despite using TableGen and SelectionDAG, there is an optimization opportunity for "SH1ADD" in MCInst form. It is done in immediate materialization where a constant node in the DAG representation is not converted to instructions until needed. In the MCTargetDesc/RISCVMatInt.cpp file, an optimization for representing immediates is provided as follows:

\begin{lstlisting}[language=C++, caption={Immediate Materialization for "SH1ADD"}]
namespace llvm::RISCVMatInt {
InstSeq generateInstSeq(int64_t Val,
                              const FeatureBitset &ActiveFeatures) {
  RISCVMatInt::InstSeq Res;
  generateInstSeqImpl(Val, ActiveFeatures, Res);
  ...

// Perform optimization with SH*ADD in the Zba extension.
  if (Res.size() > 2 && ActiveFeatures[RISCV::FeatureStdExtZba]) {
    int64_t Div = 0;
    unsigned Opc = 0;
    RISCVMatInt::InstSeq TmpSeq;
    // Select the opcode and divisor.
    if ((Val % 3) == 0 && isInt<32>(Val / 3)) {
      Div = 3;
      Opc = RISCV::SH1ADD;
    } else if ((Val % 5) == 0 && isInt<32>(Val / 5)) {
      Div = 5;
      Opc = RISCV::SH2ADD;
    } else if ((Val % 9) == 0 && isInt<32>(Val / 9)) {
      Div = 9;
      Opc = RISCV::SH3ADD;
    }
    // Build the new instruction sequence.
    if (Div > 0) {
      generateInstSeqImpl(Val / Div, ActiveFeatures, TmpSeq);
      TmpSeq.emplace_back(Opc, 0);
      if (TmpSeq.size() < Res.size())
        Res = TmpSeq;
    } else {
      // Try to use LUI+SH*ADD+ADDI.
      int64_t Hi52 = ((uint64_t)Val + 0x800ull) & ~0xfffull;
      int64_t Lo12 = SignExtend64<12>(Val);
      Div = 0;
      if (isInt<32>(Hi52 / 3) && (Hi52 % 3) == 0) {
        Div = 3;
        Opc = RISCV::SH1ADD;
      } else if (isInt<32>(Hi52 / 5) && (Hi52 % 5) == 0) {
        Div = 5;
        Opc = RISCV::SH2ADD;
      } else if (isInt<32>(Hi52 / 9) && (Hi52 % 9) == 0) {
        Div = 9;
        Opc = RISCV::SH3ADD;
      }
      // Build the new instruction sequence.
      if (Div > 0) {
        // For Val that has zero Lo12 (implies Val equals to Hi52) 
        // should has already been processed to LUI+SH*ADD
        // by previous optimization.
        generateInstSeqImpl(Hi52 / Div, ActiveFeatures, TmpSeq);
        TmpSeq.emplace_back(Opc, 0);
        TmpSeq.emplace_back(RISCV::ADDI, Lo12);
        if (TmpSeq.size() < Res.size())
          Res = TmpSeq;
      }
    }
  }

\end{lstlisting}

By using "SH1ADD" in immediate materialization, a single instruction can be used instead of two. It can be checked that in the standard testing suite, there is the following function which returns a 64-bit integer:
\begin{minipage}{\linewidth}
\begin{lstlisting}[language=llvm,style=nasm, caption={Function for Immediate Materialization}]
define i64 @PR54812() {
; RV64I-LABEL: PR54812:
; RV64I:       # %bb.0:
; RV64I-NEXT:    lui a0, 1048447
; RV64I-NEXT:    addiw a0, a0, 1407
; RV64I-NEXT:    slli a0, a0, 12
; RV64I-NEXT:    ret
;
; RV64IZBA-LABEL: PR54812:
; RV64IZBA:       # %bb.0:
; RV64IZBA-NEXT:    lui a0, 872917
; RV64IZBA-NEXT:    sh1add a0, a0, a0
; RV64IZBA-NEXT:    ret
;
  ret i64 -2158497792;
}
\end{lstlisting}
\end{minipage}
%TODO can be compared in more detail
In the FileCheck lines which are explained in Section \ref{sec:llc_test}, we can see the Assembly lines that should be emitted. It can be observed that the desired immediate can be obtained with "sh1add" in fewer instructions.

\subsection{Case Study: ROR in SelectionDAG and IR Level}

In LLVM, rotation instruction which is shifting and feeding the carry back to the shift point is captured at the IR level. To match in IR level, a new intrinsic function is defined in TableGen. TableGen is not only used in instruction selection, it is used wherever a declarative form is better suited such as in IR or MLIR levels.

\begin{lstlisting}[language=C++, caption={Funnel Shift Intrinsic Definition}]
//===-------------------- Bit Manipulation Intrinsics ---------===//
//

// None of these intrinsics accesses memory at all.
let IntrProperties = [IntrNoMem, IntrSpeculatable, IntrWillReturn]
                                                                in {
  def int_bswap: DefaultAttrsIntrinsic<[llvm_anyint_ty],
                                                 [LLVMMatchType<0>]>;
  def int_ctpop: DefaultAttrsIntrinsic<[llvm_anyint_ty],
                                                 [LLVMMatchType<0>]>;
  def int_bitreverse: DefaultAttrsIntrinsic<[llvm_anyint_ty],
                                                 [LLVMMatchType<0>]>;
  def int_fshl : DefaultAttrsIntrinsic<[llvm_anyint_ty],
      [LLVMMatchType<0>, LLVMMatchType<0>, LLVMMatchType<0>]>;
  def int_fshr : DefaultAttrsIntrinsic<[llvm_anyint_ty],
      [LLVMMatchType<0>, LLVMMatchType<0>, LLVMMatchType<0>]>;
}
\end{lstlisting}

'fshr' is defined as funnel shift right intrinsic function \cite{llvmref-fshl}. It is matched in IR optimizations by InstCombine pass.

\begin{lstlisting}[language=C++, caption={Funnel Shift Right Pattern Matching}]
/// Match UB-safe variants of the funnel shift intrinsic.
static Instruction *matchFunnelShift(Instruction &Or,
                                              InstCombinerImpl &IC){
  unsigned Width = Or.getType()->getScalarSizeInBits();

  // First, find an or'd pair of opposite shifts:
  // or (lshr ShVal0, ShAmt0), (shl ShVal1, ShAmt1)
  BinaryOperator *Or0, *Or1;
  if (!match(Or.getOperand(0), m_BinOp(Or0)) ||
      !match(Or.getOperand(1), m_BinOp(Or1)))
    return nullptr;

  Value *ShVal0, *ShVal1, *ShAmt0, *ShAmt1;
  if (!match(Or0, m_OneUse(m_LogicalShift(m_Value(ShVal0),
       m_Value(ShAmt0)))) ||
      !match(Or1, m_OneUse(m_LogicalShift(m_Value(ShVal1), 
       m_Value(ShAmt1)))) ||
      Or0->getOpcode() == Or1->getOpcode())
    return nullptr;

  // Canonicalize to or(shl(ShVal0, ShAmt0), lshr(ShVal1, ShAmt1)).
  if (Or0->getOpcode() == BinaryOperator::LShr) {
    std::swap(Or0, Or1);
    std::swap(ShVal0, ShVal1);
    std::swap(ShAmt0, ShAmt1);
  }

  // Match the shift amount operands for a funnel shift pattern. 
  // This always matches a subtraction on the R operand.
  auto matchShiftAmount = [&](Value *L, Value *R, unsigned Width) 
                                                       -> Value * {
    // Check for constant shift amounts that sum to the bitwidth.
    ...

    return nullptr;
  };

  Value *ShAmt = matchShiftAmount(ShAmt0, ShAmt1, Width);
  bool IsFshl = true; // Sub on LSHR.
  if (!ShAmt) {
    ShAmt = matchShiftAmount(ShAmt1, ShAmt0, Width);
    IsFshl = false; // Sub on SHL.
  }
  if (!ShAmt)
    return nullptr;

  Intrinsic::ID IID = IsFshl ? Intrinsic::fshl : Intrinsic::fshr;
  Function *F = Intrinsic::getDeclaration(Or.getModule(), IID,
                                                   Or.getType());
  return CallInst::Create(F, {ShVal0, ShVal1, ShAmt});
}
\end{lstlisting}

The pattern matching API is provided by IR/PatternMatch.h file. After the pattern for rotation and the shift amount are matched the corresponding intrinsic function is called. 

The function is called in OR visiting function, so the root of the pattern is OR:

\begin{lstlisting}[language=C++, caption={Funnel Shift Right Pattern Function Called}]
Instruction *InstCombinerImpl::visitOr(BinaryOperator &I) {
  ...
  if (Instruction *Funnel = matchFunnelShift(I, *this))
    return Funnel;
  ...
  return nullptr;
}


\end{lstlisting}

The intrinsic function is converted to ROTR SDnode in the general SelectionDAGBuilder.cpp file.
\begin{lstlisting}[language=C++, caption={Funnel Shift Intrinsic converted to ROTL}]
/// Lower the call to the specified intrinsic function.
void SelectionDAGBuilder::visitIntrinsicCall(const CallInst &I,
                                             unsigned Intrinsic) {
  const TargetLowering &TLI = DAG.getTargetLoweringInfo();
  SDLoc sdl = getCurSDLoc();
  DebugLoc dl = getCurDebugLoc();
  SDValue Res;

  SDNodeFlags Flags;
  if (auto *FPOp = dyn_cast<FPMathOperator>(&I))
    Flags.copyFMF(*FPOp);

  switch (Intrinsic) {
  default:
    // By default, turn this into a target intrinsic node.
    visitTargetIntrinsic(I, Intrinsic);
    return;
  ...

  case Intrinsic::fshl:
  case Intrinsic::fshr: {
    bool IsFSHL = Intrinsic == Intrinsic::fshl;
    SDValue X = getValue(I.getArgOperand(0));
    SDValue Y = getValue(I.getArgOperand(1));
    SDValue Z = getValue(I.getArgOperand(2));
    EVT VT = X.getValueType();

    if (X == Y) {
      auto RotateOpcode = IsFSHL ? ISD::ROTL : ISD::ROTR;
      setValue(&I, DAG.getNode(RotateOpcode, sdl, VT, X, Z));
    } else {
      auto FunnelOpcode = IsFSHL ? ISD::FSHL : ISD::FSHR;
      setValue(&I, DAG.getNode(FunnelOpcode, sdl, VT, X, Y, Z));
    }
    return;
  }
  ...
}

\end{lstlisting}


The intrinsic functions can be defined as target-specific or target independent. "fshl" is a general intrinsic function and targets can either expand it by replacing it with its equivalent instructions or lower it directly to an instruction by legalizing it. RISC-V Zbb extension supports bitwise rotation so LLVM has the extension's implementation in the source. In RISCVISelLowering.cpp file the legalization of "ROTR" is managed regarding whether the extension is enabled or disabled.

\begin{lstlisting}[language=C++, caption={ROTR Legalization Conditional}]
RISCVTargetLowering::RISCVTargetLowering(const TargetMachine &TM,
                                         const RISCVSubtarget &STI)
    : TargetLowering(TM), Subtarget(STI) {
    ...
  if (Subtarget.hasStdExtZbb() || Subtarget.hasStdExtZbkb() ||
      Subtarget.hasVendorXTHeadBb()) {
    if (Subtarget.is64Bit())
      setOperationAction({ISD::ROTL, ISD::ROTR}, MVT::i32, Custom);
  } else {
    setOperationAction({ISD::ROTL, ISD::ROTR}, XLenVT, Expand);
  }
  ...
}
\end{lstlisting}

The "Custom" action is defined in TableGen in RISCVInstrInfoZb.td file as well as instruction encodings.

\begin{lstlisting}[language=C++, caption={ROR Encodings and Pattern}]
def riscv_rolw   : SDNode<"RISCVISD::ROLW",   SDT_RISCVIntBinOpW>;
def riscv_rorw   : SDNode<"RISCVISD::RORW",   SDT_RISCVIntBinOpW>;
...
let Predicates = [HasStdExtZbbOrZbkb] in {
def ROL   : ALU_rr<0b0110000, 0b001, "rol">,
            Sched<[WriteRotateReg, ReadRotateReg, ReadRotateReg]>;
def ROR   : ALU_rr<0b0110000, 0b101, "ror">,
            Sched<[WriteRotateReg, ReadRotateReg, ReadRotateReg]>;

def RORI  : RVBShift_ri<0b01100, 0b101, OPC_OP_IMM, "rori">,
            Sched<[WriteRotateImm, ReadRotateImm]>;
} // Predicates = [HasStdExtZbbOrZbkb]
...
let Predicates = [HasStdExtZbbOrZbkb] in {
def : PatGprGpr<shiftop<rotl>, ROL>;
def : PatGprGpr<shiftop<rotr>, ROR>;

def : PatGprImm<rotr, RORI, uimmlog2xlen>;
// There's no encoding for roli in the the 'B' extension as it can be
// implemented with rori by negating the immediate.
def : Pat<(rotl GPR:$rs1, uimmlog2xlen:$shamt),
          (RORI GPR:$rs1, (ImmSubFromXLen uimmlog2xlen:$shamt))>;
} // Predicates = [HasStdExtZbbOrZbkb]
\end{lstlisting}

As you can see when the pattern match logic is lifted up to the IR level the modifications in Instruction Selection are straightforward to implement. 



