% ----------------------------------------------------------------- %
%          ITU LaTeX Undergraduate Interim Report Template	         %
%                                                                   %
%		         		           Version 1.0                       	        %
% ----------------------------------------------------------------- %

% ----------------------------------------------------------------- %
% Prepared by S. Baris Ozturk                                       %
% Based on the Thesis Template of ITU Informatics Institute	        %
%                                                                   %
% Informatics Institute, ITU Ayazaga Campus                         %
% Maslak-34469, İstanbul, Turkey									                           %
% http://www.be.itu.edu.tr											                               %
% ----------------------------------------------------------------- %

% ----------------------------------------------------------------- %
% Updated voluntarily by	  										                               %
% S. Baris Ozturk   : ozturksb@itu.edu.tr OR salihbaris@gmail.com   %
% E. Baris Ondes    : ondes@itu.edu.tr 	  OR ondesalt@pm.me			      % 
% Tamer Sener	      : senerta@itu.edu.tr  OR tamersener@pm.me	      %
% Berkan Kacmaz	    : kacmazb@itu.edu.tr  OR kacmazberkan0@gmail.com%
% O. Ozgun Altunkaya: altunkayao@itu.edu.tr OR ozgun@pm.me 			      %
% ----------------------------------------------------------------- %

% ----------------------------------------------------------------- %
% Please visit the following page for the current version:		       	%
% https://github.com/ondes/Template-Latex-ITU-Interim-Report	   	   %
% ----------------------------------------------------------------- %

% ----------------------------------------------------------------- %
% documentclass arguments:             							    %
% [onluarkali,tekyonlu],[turkce,ingilizce],						    %
% [lisans],					                						%
% [elektrikelektronik]			   									%
% Sample use: \documentclass[onluarkali,ingilizce,lisans,    		%
% elektrikelektronik]{interimreport_itu}					  		%
% onluarkali: twoside                     			  			   	%
% tekyonlu: oneside						 						   	%
% turkce: turkish						 						   	%
% ingilizce: english											   	%
% elektrikelektronik: electrical and electronics				    %
% ----------------------------------------------------------------- %

\documentclass[onluarkali,ingilizce,lisans,elektrikelektronik]{interimreport_itu}

% ----------------------------------------------------------------- %
% Pay attention to letter case. The structure is case sensitive     %
% {Student Name}{SURNAME} means first letter of name is capital     %
% and that all letters of surname are capital.                      %
% ----------------------------------------------------------------- %

% ----------------------------------------------------------------- %
% No titles, only Name and SURNAME must be written.      	        %
% ----------------------------------------------------------------- %

\yazarbir{Mehmet Eymen}{ÜNAY}
\ogrencinobir{040190218}
\yazariki{Bora}{İNAN}
\ogrencinoiki{040190205}
\yazaruc{Emrecan}{YİĞİT}
\ogrencinouc{040190205}

\tezyoneticisi{Dr. Tankut AKGÜL}
 
% ----------------------------------------------------------------- %
% For below, the first argument for Turkish, the second is for      %
% English.       						   						    %
% ----------------------------------------------------------------- %

% ----------------------------------------------------------------- %
% Only the first letters of words must be capital.		     	    %
% ----------------------------------------------------------------- %

\anabilimdali{Elektrik Mühendisliği Anabilim Dalı}{Electronics and Communication Engineering}
\programi{Elektrik Mühendisliği Programı}{Electrical Engineering Programme}

% ----------------------------------------------------------------- %
% The interim report submission date in upper and lower case letters%
% ----------------------------------------------------------------- %

\tarihKucuk{Ara Rapor Tarihi}{January 2023}

% ----------------------------------------------------------------- %
% Thesis title Turkish or English                                   %
% ----------------------------------------------------------------- %

\baslik{TEZ BAŞLIĞI BURAYA GELİR}
{GEREKLİ İSE İKİNCİ SATIR}
{GEREKLİ İSE ÜÇÜNCÜ SATIR, ÜÇ SATIRA SIĞDIRINIZ}

\title{SUPPORTING CUSTOM INSTRUCTIONS WITH THE LLVM COMPILER } % First letters can be capitalized 
{FOR RISC-V PROCESSOR}
{}

% ----------------------------------------------------------------- %
% Packages and definitions to be used are given in the following    %
% ----------------------------------------------------------------- %

\input{defs}                                           

\begin{document}

% ----------------------------------------------------------------- %
% Also form the below files. All corresponding inputs should point  %
% to a file. They are provided inside the archive.                  %
% ----------------------------------------------------------------- %

%%%%%%%%%%%%%%%%%%%%%%%%%%%%%%%%%%%%%%%%%%%%%%%%%%%%%%%%%%%%%%%%%
% !TEX root = interimreport.tex
\chapter{INTRODUCTION}\label{Ch1}
%%%%%%%%%%%%%%%%%%%%%%%%%%%%%%%%%%%%%%%%%%%%%%%%%%%%%%%%%%%%%%%%%
%%TODO Better Intro
%Key points:
%%ASIP are more efficient, fast 
%%Compilers are needed to be modified to run software already developed
%%Without compiler support ASIP becomes limited for acceleration

% give hints about the upcoming chapters 
%%compilers, llvm, riscv, ascon, llvm backend, custom instr, pattern matching and maybe testing
Recent advances and studies on Integrated Circuits (IC) caused technology to produce application-specific circuits for various areas of usage. Extensions for the open source processor architectures became a part of the industrial development. More custom accelerators are developed, especially with RISC-V open and modular instruction set architecture (ISA). Hardware accelerators have the promise of being fast and efficient.

However, loading new abilities to an extended processor comes with a problem. Programming languages and their compilers are developed for common architectures. A compiler targeting standard ISA will not produce the custom instructions for the accelerator. A compiler modification is needed to be able to introduce the accelerator to the high-level languages. In this thesis, we show various ways to approach the problem and present the best practices for it. 

%TODO Talk about LLVM

%TODO mention ascon for sbox 
For the research, several accelerators with specific custom instructions are targeted \cite{Sairoglu, eryilmaz}. Instructions which are targeted to hardware are SHLXOR, RORI and S-box. The encodings and instruction operations were mostly designed by the hardware developers. The process of required compiler modifications for SHLXOR and RORI are demonstrated in Sections \ref{sec:shlxor} and \ref{sec:rori}. S-box instruction, due to its non-linearity, was a complicated instruction to characterize. It is a good example that not every instruction can be added in a similar process and instruction-specific design can be required. 
%TODO talk about instruction design and riscv, maybe in separate paragraph after the one below
Also similar to the design of ISAs, instructions should be designed by considering both hardware and software. 

S-box instruction is analyzed from several aspects. Firstly, the Intermediate Representation (IR) optimizations it gets through are demonstrated in depth in Section \ref{sbox-case}. Secondly, the limitations of TableGen which was a sufficient system for the previous instructions, are discussed and C++ pattern matching is explained in Section \ref{sec:cpp}. 
Thirdly, pattern matching in IR and MCInst level are discussed in Section \ref{sec:patmatchdisc}. Finally, we proposed two new instructions that can be implemented in hardware that can accelerate S-box operation as LXR and NAXOR. A simplified version of LXR which has independent Load addresses is demonstrated in Section \ref{sec:lxr}. The S-box case where the load addresses are dependent is presented in C++ pattern matching in Section \ref{sec:cpp}. The second proposed instruction, NAXOR, is shown in Section \ref{sec:naxor}.

In conjunction with LXR and NAXOR which do not have target hardware, MLA instruction is also presented without target hardware. MLA is discussed in detail in Chapter \ref{Ch4} where it is traced from the C code to Assembly in steps of compilation and Section \ref{sec:MLA_add_section} where its support was added with TableGen. 

%TODO Pattern Matching 

\section{Purpose of Project}
Application-Specific Instruction Set Processors (ASIP) are becoming more popular with the development of embedded systems. The specialization of the core causes a tradeoff between flexibility and performance. For special purposes, using ASIPs increases efficiency, however, we can program a custom ASIP only by using assembly instructions that we defined. Programming custom processors with assembly language is not a preferred way of coding. We are also not able to use high-level languages because compiling tools are designed for common architectures with certain instructions. The ability to add custom instructions to compilers will enable us to make more use of custom hardware designs.

ASIPs are feasible for all application-specific embedded systems like consumer, industrial, automotive, home appliances, cryptology, medical, telecommunication, commercial, aerospace, and military applications. The custom back-end that we will design under the supervision of Dr. Tankut Akgül, is going to serve the processor designed by Prof. Dr. Sıddıka Berna Örs Yalçın’s research team. When the project is completed, Prof. Yalçın is going to be able to produce the assembly codes that are compatible with the processor’s extended instruction set in addition to RISC-V.


Prof. Yalçın and her team are designing application-specific instruction set processors. The purpose of this project is to create a compiler back-end for a processor that supports custom instructions on top of RISC-V instructions. This compiler is going to help to program the custom processor by using high-level languages. Existing RISC-V compilers are not able to produce efficient assembly codes for ASIPs. Therefore a need arose for a compiler back-end.
The main reason for choosing this project is that we wanted to meet an actual need for a critical existing problem. The project has the potential to be the bridge between hardware and software of custom hardware projects in research, enabling them to be candidates for production use cases. 
                                       
%%%%%%%%%%%%%%%%%%%%%%%%%%%%%%%%%%%%%%%%%%%%%%%%%%%%%%%%%%%%%%%%%
% !TEX root = interimreport.tex
\clearpage
\chapter{BASICS OF A COMPILER}\label{Ch2}
%%%%%%%%%%%%%%%%%%%%%%%%%%%%%%%%%%%%%%%%%%%%%%%%%%%%%%%%%%%%%%%%%
A compiler is a software that converts source code written in a high-level programming language into machine code appropriate for a particular computer architecture.
There are different stages of a compiler but they can be grouped into two main parts such as “Front-End” and “Back-End”. %\cite{phasesofcomp}. 
These parts of the compiler are also called the analysis and synthesis parts of the compiler. The analysis stage separates the source program into its individual components and applies a grammatical structure to them. The source code is then represented in an intermediate stage using this structure. The synthesis phase creates the final target program by using the intermediate representation. We can think of the compilation process as a series of phases, each of which takes the source program and transforms it into another representation \cite{compileralfredaho}. These phases can be seen in Figure \ref{fig:comp_stages}.
\begin{figure}
    \centering
    \includegraphics[scale=0.25]{basics_of_compiler/comp_stages.jpeg}
    \caption{Compiler Stages}
    \label{fig:comp_stages}
\end{figure}


\section{Front-End}
%TODO mention parsing here in a paragraph

\subsection{Lexical Analysis}
The compiler breaks down the source code into smaller units called lexemes, which are pieces of code that correspond to specific patterns in the code. These lexemes are then converted into tokens that can be used for syntax and semantic analyses.

\subsection{Syntax Analysis}
The compiler checks that the code follows the proper syntax for the programming language it is written in. This process is also called parsing. As part of this step, the compiler often creates abstract syntax trees in order to represent the logical structure of different parts of the code.

\subsection{Semantic Analysis}
The compiler checks that the code makes logical sense, going beyond syntax analysis by ensuring that the code is correct. For example, the compiler might check that variables have been declared correctly and given the appropriate data types. This process is known as semantic analysis.

\subsection{IR Code Generation}
After the source code has been analyzed for lexemes, syntax, and semantics, the compiler creates an intermediate representation (IR) of the code.
This intermediate code is going to be converted to machine code in the last two phases. These two phases are platform-dependent, meaning they are specific to a particular hardware architecture but the previous phases were not. Therefore, to create a new compiler, it is not necessary to start from scratch. Instead, it is possible to use the intermediate code from an existing compiler and build the final stages of the process for a specific platform. Because of that, we are interested in the back-end part for our project.

\section{Middle-End}

\subsection{Optimization}
The intermediate code is prepared for the final code generation step. This process does not change the meaning or functionality of the code, but it can make the program run faster and more efficiently.
A directed acyclic graph (DAG) used in the compiler design process might represent the dependencies between different instructions in the IR code, such as the order in which those instructions need to be executed or the data dependencies between them. The DAG can be used by the compiler to identify opportunities for optimization, such as removing unnecessary instructions, combining some of them, or rearranging the order of execution to reduce the number of resources required by the code.
Static single assignment (SSA) is also an important part of optimization. It is a technique that is used to organize the intermediate representation in a way that ensures each variable is assigned a value only once and that each variable is defined before it is used.
%TODO we should mention SSA form as well

\section{Back-End}



\subsection{Target Code Generation}
The target code generator is the final stage of the compilation process, and its main function is to convert the optimized code into a form that the machine can understand. The optimized code is turned into a relocatable machine code. The relocatable machine code is the input to the linker and loader, which are responsible for combining the code with other necessary resources and preparing it for execution %\cite{tutorialcomp}. 
Target code generation can be divided into different parts:

\subsection{Instruction Selection} 
IR is the input of the code generation step, and it maps the IR into the target machine’s instruction set. There may be multiple ways for converting one representation, so the code generator tries to select the most suitable instructions.

\subsection{Register Allocation}
There may be many different variables/values in a program. The code generator decides which registers to use to keep these values.

\subsection{Instruction Scheduling}
The code generator determines the sequence in which instructions will be executed and creates schedules for the execution of those instructions.


%%%%%%%%%%%%%%%%%%%%%%%%%%%%%%%%%%%%%%%%%%%%%%%%%%%%%%%%%%%%%%%%%
% !TEX root = interimreport.tex
\clearpage
\chapter{THE LLVM COMPILER}\label{ch:Ch3}
%%%%%%%%%%%%%%%%%%%%%%%%%%%%%%%%%%%%%%%%%%%%%%%%%%%%%%%%%%%%%%%%%
%\vspace*{-12pt} % If no text above section, use this vspacing to lift the whole part to the proper starting point - SBÖ

LLVM is a collection of modular and flexible compiler and toolchain software that can be used to build a wide variety of compilers and other tools. LLVM compilers are organized into a set of libraries that implement the parts of a compiler. There are different front-end libraries a for every language and different back-end libraries for every architecture. There is only one common intermediate representation optimizer that connects specific front-end and back-end.

\begin{figure}
    \centering
    \includegraphics{the_llvm_compiler/llvm_diagram.png}
    \caption{Front-end and Back-end libraries connected by LLVM}
    \label{fig:llvm_diagram}
\end{figure}

LLVM IR is the common representation of languages. Various LLVM front-ends translate related languages into IR. Related back-end compiles IR into assembly according to the target hardware. This structure helps to increase flexibility between front-ends and back-ends. With this structure, we are able to have compilers for every combination of M source codes and N targets with M front-end and N back-end instead of M*N compilers. In our case, we do not have to struggle with the frontend also our backend will be working for every source code of LLVM because of this benefit. The front-end we use in the developing process will be clang which is the LLVM C/C++ front-end\cite{clang}.

\section{Parts of the Clang Frontend}

\subsection{Clang Lex Library}
Clang Lex Library is a typical lexer implemented as finite state machines that read source code one character at a time and transition between different states based on the characters read. The Clang lexer, which is a front-end compiler for the C, C++, and Objective-C programming languages, uses this approach to filter out comments and white space, recognize and tokenize language elements such as keywords, identifiers, and operators, and handle escape sequences and string literals. The implementation files of the Clang lexer can be found in the llvm-project/clang/lib/lex directory within the LLVM infrastructure.
	

\subsection{Clang Parse Library}

Clang Parse Library is the parser that takes the tokens produced by the lexer and constructs an abstract syntax tree (AST) to represent the structure and meaning of the source code. The Clang parser checks the source code for proper syntax and resolves symbols and identifiers. It also performs type-checking to ensure the source code follows the rules of the programming language. It creates the AST, a tree-like structure, that represents the source code in a way that is easily processed by the compiler. The implementation files of the Clang lexer can be found in the llvm-project/clang/lib/parse directory within the LLVM infrastructure.

\subsection{Clang Sema Library}

Clang Sema Library is a semantic analyzer that involves examining the meaning and context of the source code in a program. In Clang, semantic analysis is a phase in the compilation process that analyzes the abstract syntax tree (AST) generated by the parser to verify that the source code conforms to the rules of the programming language and is properly constructed. Semantic analysis performs various checks and transformations on the AST to ensure the source code is correct. The implementation files of the Clang semantic analyzer can be found in the llvm-project/clang/lib/sema directory within the LLVM infrastructure.

\subsection{Clang CodeGen Library}

Clang CodeGen is the code generation library that takes the abstract syntax tree which is generated by the parser and corrected by the semantic analyzer as input. It generates the intermediate representation code and produces .ll file which will be used in the backend. The implementation files of the Clang lexer can be found in the llvm-project/clang/lib/CodeGen directory within the LLVM infrastructure. 

\section{LLVM IR Optimizer}
LLVM Optimizer is a common optimization medium used for every possible source-target combination of a compiler. It takes the output file of CodeGen as input and runs three types of passes:

\begin{enumerate}
    \item Analysis passes: These passes analyze the IR and collect information about the IR without modifying the IR. 
    \item Transformation passes: These passes modify the IR by using the information gathered from Analysis passes. The optimizations are the product of these transformations. 
    \item Utility passes: These passes are used to perform tasks such as printing the IR or verifying the IR. 
\end{enumerate}

The output of the optimizer becomes the input for the target back-end which lowers the LLVM IR to the target Assembly. As the generated LLVM IR at the end of the optimizations is the object of pattern matching and assembly support for any custom instruction, it is a critical part of the design process.

Clang can be invoked with optimization levels deciding which optimization passes are going to be run. The optimisations can target speed or code size. Speed optimising options range from "-O1" to "-O3". "-O2" enables most of the optimizations. "-O3" enables optimisations that can increase the compile time and generate larger code. Main code optimising options are "-Os" and "-Oz". "-Os" is similar to "-O2" but runs extra optimizations to reduce code size. "-Oz" runs more code reducing optimisations compared to "-Os" and is similar to "-O2" again \cite{clangCommands}.
Caution must be taken as when no arguments are given to Clang, at the time of writing, Clang uses the "-O0" optimization level. Implementing pattern matching on unoptimised LLVM IR is not feasible for several reasons. Firstly, the IR is more sensitive to changes in the frontend. Changing the code style in frontend can cause CodeGen to produce a slightly different IR which makes it less predictable. Secondly, the code size can be too large with redundant code which makes pattern matching large instructions cumbersome. We recommend using "-O2" or "-Os" optimization levels while developing instruction selection patterns. 




LLVM optimizer can be performed with opt [options] [filename] command\cite{optimizer}.

\section{Stages of the LLVM RISC-V Back-end}
LLVM RISC-V Backend is responsible for compiling optimized IR down to RISC-V assembly or object code. LLVM Backend consists of libraries for the code generation steps\cite{llvmbackend}.

\subsection{Instruction Selection}
SelectionDAG is the instruction selector of LLVM backends which is responsible for selecting the appropriate RISC-V instructions for a given intermediate representation instruction. It takes the target-independent LLVM code as input and generates the target-dependent DAG of instructions. SelectionDAG is the most important part of the compiler for us since we will be dealing with adding new instructions to the RISC-V Backend. 

\subsection{Instruction Selection}
SelectionDAG is the instruction selector of LLVM backends which is responsible for selecting the appropriate RISC-V instructions for a given intermediate representation instruction. It takes the target-independent LLVM code as input and generates the target-dependent DAG of instructions. SelectionDAG is the most important part of the compiler for us since we will be dealing with adding new instructions to the RISC-V Backend. 

\subsubsection{SelectionDAG construction}
After IR generating is done, SelectionDAG gets the optimized ir and converts it into target independent SelectionDAG representation. 

\subsubsection{SelectionDAG legalization}
SelectionDAG is a target dependent representation after the initial stage of the instruction selection. Before creating a target specific code, SelectionDAG checks if the DAG is legal because the constructed DAG may include incompatible instructions and data types to the target architecture. SelectionDAG legalizes the illegal DAG into a supported form. 

\subsubsection{SelectionDAG optimization}
SelectionDAG is in an optimizable form after legalization because legalization phase may create unnecessary DAG nodes and the reducable nodes are not combined yet. SelectionDAG optimizer minimzes the DAG nodes before creating the target spesific instructions.

\subsubsection{SelectionDAG target dependent instruction selection}
At the last phase of the instruction selection SelectionDAG selects the suitable instructions to the target architecture. SelectionDAG uses the relevant tablegen target description (.td) files to match the patterns and creates the target specific instructions.

\subsection{Scheduling and Formation}
Scheduling is the phase of assigning an order to the DAG form of RISC-V instructions. The formation phase is responsible for converting the DAG into a list of machine instructions.  

\subsection{SSA-based Machine Code Optimizations}
LLVM uses static single assignment (SSA) based optimizations before register allocation. SSA optimizations ensure that each variable is assigned and defined only once before it is used. 


\subsection{Register Allocation}
The register allocator of RISC-V back-end is responsible for assigning physical registers to virtual registers in the IR. Each target has specific register count and order. Register allocator maps the registers by taking the RISC-V architecture registers into account. It uses the relevant TargetRegisterInfo, and MachineOperand classes. Register allocation will play a key role in our project while we manipulating the implementation of the selected instructions.

\subsection{Prolog/Epilog Code Insertion}
Prolog and epilog code insertion is another optimization phase which is responsible for frame-pointer elimination and stack packing.


\subsection{Code Emission}
Code emission stage is responsible for lowering the code generator abstractions down to the MC layer abstractions. It takes the assembly as input and creates the final RISC-V machine codes. 

\subsection{Linking}
LLD is the LLVM linker library that is responsible for combining multiple object files into a single executable file. LLD is invoked after the code emission and generates a file by resolving symbol references, adjusting addresses, and performing other tasks as necessary.

%%%%%%%%%%%%%%%%%%%%%%%%%%%%%%%%%%%%%%%%%%%%%%%%%%%%%%%%%%%%%%%%%
% !TEX root = interimreport.tex

\clearpage
\chapter{PATH OF AN INSTRUCTION}\label{Ch4}
%%%%%%%%%%%%%%%%%%%%%%%%%%%%%%%%%%%%%%%%%%%%%%%%%%%%%%%%%%%%%%%%%
In this chapter, the path of an instruction will be demonstrated and the corresponding DAG input of the most critical phases of SelectionDAG will be shown. We selected the input program as a function that performs multiplication and addition. This was our litmus test code used while adding MLA (Multiply and Add) instruction to the LLVM backend with TableGen. We explained our addition of MLA instruction thoroughly in Section \ref{sec:MLA_add_section}. 

\begin{lstlisting}[language=C, caption=madd.c program]
int a,b,c;
void maddFunc() {
	a = 3;
	b = 103;
	
	c = 127;
	a = a * b + c;
}
\end{lstlisting}

\section{Clang AST}
You can see the Abstract Syntax Tree (AST) produced by Clang in Figure \ref{fig:clang_ast}. The AST consists of an expression tree with three levels. At the highest level there is an expression tree of multiplication between variables 'a' and 'b'. This expression tree's result becomes an argument for another expression tree with the addition operator. The second argument at this addition subtree is the variable 'c'. The expression tree at the root has assignment as an operator. The first argument to this tree is 'a' and the second argument is the result of multiplication and addition. Figure \ref{fig:ast_mla} shows the entire expression tree with the operations multiply and add. 

\begin{figure}
    \centering
    \includegraphics[width=\textwidth]{path_instruction/madd_clang_ast_cropped.png}
    \caption{AST generated by Clang}
    \label{fig:clang_ast}
\end{figure}
\begin{figure}
    \centering
\begin{tikzpicture}

    \node {=} [sibling distance = 2.5cm]
    child {node {+}
    child {node {*}
    child {node {\bf a}}
    child {node {\bf b}}}
    child {node {\bf c}}}
    child {node {\bf a}};
\end{tikzpicture}
    \caption{AST of MLA operation}
    \label{fig:ast_mla}
\end{figure}
\section{LLVM IR}
Clang CodeGen produces LLVM IR with the AST as the input. Figure \ref{fig:llvm_ir} shows the produced LLVM IR. The optimized LLVM IR is the input to SelectionDAG to generate target-specific instructions.  
\par
\begin{figure}
    \centering
    \includegraphics[width=0.6\textwidth]{path_instruction/madd_ll.png}
    \caption{LLVM IR file generated at the output of Clang}
    \label{fig:llvm_ir}
\end{figure}

\section{SelectionDAG}
 Input DAGs to SelectionDAG's passes will be demonstrated so on. The following phases will be demonstrated:
\begin{enumerate}
    \item First Optimization
    \item Legalization
    \item Second Optimization
    \item Instruction Selection
    \item Instruction Scheduling 
    \item Register Allocation
\end{enumerate}
\subsection{First Optimization Pass}
Figure \ref{fig:combine1} shows the DAG before the first optimization pass. It is the direct translation of LLVM IR to DAG form. After optimization, redundant nodes will be removed such as "Constant<0>" node.
\begin{figure}
    \centering
    \includegraphics[width=0.9\textwidth]{path_instruction/madd_dag_combine1.png}
    \caption{DAG before first optimization pass}
    \label{fig:combine1}
\end{figure}

Figure \ref{fig:legalize} shows the DAG before legalization. The first optimization took place by removing nodes that do not contribute to the DAG. However, the instructions are not, in LLVM terms, "legal" as these general Machine Instructions do not map directly to every target's instructions.  

\subsection{Instruction Legalization}

\begin{figure}
    \centering
    \includegraphics[width=0.9\textwidth]{path_instruction/madd_dag_legalize.png}
    \caption{DAG before Legalization}
    \label{fig:legalize}
\end{figure}

Figure \ref{fig:combine2} shows the DAG before the second optimization pass. The DAG is legalized by introducing RISCVISD::ADD\_LO and RISCVISD::HI nodes. These SelectionDAG nodes act as flags to give target-specific information to target-independent algorithms. These definitions are introduced at lib/Target/RISCV/RISCVISelLowering.h file \cite{riscvIselh}. It is the RISCV DAG lowering interface. 
\par
According to the interface file, RISCVISD::ADD\_LO is meant to add Lo 12 bits from an address and to be replaced by ADDI (Add Immediate) at Instruction Selection. Similarly, RISCVISD::HI is meant to get Hi 20 bits from an address and to be replaced by LUI (Load Upper Immediate). With a legalized DAG the second optimization pass begins.
\subsection{Second Optimization Pass}
\begin{figure}
    \centering
    \includegraphics[width=0.9\textwidth]{path_instruction/madd_dag_combine2.png}
    \caption{DAG before the second optimization}
    \label{fig:combine2}
\end{figure}

Figure \ref{fig:isel} shows the DAG before the Instruction Selection phase. Comparison of Figure \ref{fig:combine2} and \ref{fig:isel} indicates that the second optimization did not change the DAG. This may be due to the reason that the subgraphs including the legalized nodes are not complex enough as the input C code is minimal.
\par
The DAG nodes up until Instruction Selection are instances of SDNode class which are target-independent nodes.
\begin{figure}
    \centering
    \includegraphics[width=0.9\textwidth]{path_instruction/madd_dag_isel.png}
    \caption{DAG before Instruction Selection}
    \label{fig:isel}
\end{figure}

\subsection{Instruction Selection}
Figure \ref{fig:dag_sched} shows the DAG before the Instruction Scheduling phase. You can see that the instructions are selected according to the RISC-V target. SDNode class nodes are replaced by MachineSDNode class nodes which are target-specific.
\par
RISCVISD nodes are replaced by their counterparts. The general Load and Store instructions are replaced by their type-aware corresponding LW (Load Word) and SW (Store Word) instructions. Most importantly the MLA instruction is selected replacing the subgraph of 'mul' and 'add' LLVM instructions. 
\par
We defined MLA instruction's DAG pattern as in the subgraph. The instruction selection phase took it as a reference, detected the pattern inside the global DAG and used it to place the MLA node. The addition process is explained more thoroughly in Section \ref{MLA_add_section}.

\begin{figure}
    \centering
    \includegraphics[width=0.9\textwidth]{path_instruction/madd_dag_sched.png}
    \caption{DAG before Instruction Scheduling}
    \label{fig:dag_sched}
\end{figure}

\subsection{Instruction Scheduling}
The DAG is transformed into a target-specific DAG with the result of legalization and selection phases. However, to generate a linear byte sequence, the DAG must be flattened. The instruction scheduling phase gets the DAG and linearises it according to the dependency graph of nodes. The scheduling dependency can be seen in Figure \ref{fig:sunit}.

\begin{figure}
    \centering
    \includegraphics[width=0.9\textwidth]{path_instruction/madd_dag_sunit.png}
    \caption{Scheduling Dependency Graph}
    \label{fig:sunit}
\end{figure}


\subsection{Machine Instruction in SSA Form}
The generated Machine Instruction as a result of scheduling is shown in Figure \ref{fig:mc_inst}. Because register allocation is not yet performed, the instructions are in SSA (Static Single Assignment) form. In SSA form, virtual registers are considered to be infinite unless some specific registers have to be used. In this case, '\$x0' is mentioned with ADDI instructions as they are hardwired zero in RISC-V. 

\begin{figure}
    \centering
    \includegraphics[width=0.9\textwidth]{path_instruction/madd_MachineInstruction.png}
    \caption{Machine Instruction before Register Allocation}
    \label{fig:mc_inst}
\end{figure}

\clearpage
\section{Machine Code Instruction}
After register allocation, Machine Code Instruction (MCInst) representation of the code is created. MCInst can be thought of as an intermediate representation of the lower-level code. It can be used to produce both an object file and an Assembly file. The generated Assembly is presented below:
\begin{lstlisting}[ caption=madd.s Assembly Output]
    maddFunc:                               # @maddFunc
# %bb.0:
	addi	sp, sp, -16
.Ltmp0:
	sw	ra, 12(sp)                      # 4-byte Folded Spill
	sw	s0, 8(sp)                       # 4-byte Folded Spill
	addi	s0, sp, 16
	lui	a0, %hi(a)
	li	a1, 3
	sw	a1, %lo(a)(a0)
	lui	a1, %hi(b)
	li	a2, 103
	sw	a2, %lo(b)(a1)
	lui	a2, %hi(c)
	li	a3, 127
	sw	a3, %lo(c)(a2)
	lw	a3, %lo(a)(a0)
	lw	a1, %lo(b)(a1)
	lw	a2, %lo(c)(a2)
	mla	a1, a3, a1 ,a2
	sw	a1, %lo(a)(a0)
	lw	ra, 12(sp)                      # 4-byte Folded Reload
	lw	s0, 8(sp)                       # 4-byte Folded Reload
	addi	sp, sp, 16
	ret
\end{lstlisting}

%%%%%%%%%%%%%%%%%%%%%%%%%%%%%%%%%%%%%%%%%%%%%%%%%%%%%%%%%%%%%%%%%
% !TEX root = interimreport.tex
\clearpage
\chapter{ADDING CUSTOM INSTRUCTIONS}\label{ch:custom_instr}
%%%%%%%%%%%%%%%%%%%%%%%%%%%%%%%%%%%%%%%%%%%%%%%%%%%%%%%%%%%%%%%%%
The instruction selection system we focused on at the back-end of the LLVM compiler is SelectionDAG among FastISel and GlobalIsel. SelectionDAG is the most mature Instruction Selection framework with more target support. However, in the near future, it is worth considering GlobalISel as it is developed recently as an alternative to SelectionDAG. The reasons to replace it are to make it faster, smaller, and more open to low-level optimizations.
\section{TableGen Reference}
TableGen is a domain-specific language used in LLVM backend side to generate CPP header files. The purpose it serves is that it reduces redundancy of instruction declaration code which can be common to numerous architectures with minor differences. To maintain and scale the framework the minor differences are implemented level by level at a series of inheritance operations between TableGen classes. 

\par
 LLVM Static Compiler, LLC, is responsible for converting LLVM IR to Assembly code. To add new instructions LLC is recompiled and its internals change. While the compilation operation of the LLC program, TableGen records are created which declare every instruction’s encoding and describe its features. Referring to the records, directed acyclic graphs (DAG) are used in the process of instruction selection. DAG is a graph structure which has no cycles and has directions on the edges. Operations or functions are represented as nodes in the DAG. They are critical parts of declaring the logic or pattern of the new instruction. 
\par

The operations represented on the DAG can be LLVM intrinsics as well as instructions. LLVM instructions resemble conventional assembly instructions, in contrast, LLVM intrinsics have higher level abstraction depending on their functionality. Their instruction generation may vary depending on the target hardware. It is possible to define a new complicated instruction either by combining simple LLVM instructions and higher level intrinsics in DAG level or by creating a new LLVM intrinsic which gets created at the Intermediate Level of the compilation process.

\section{RISC-V TableGen Classes}
The most general instruction class used for every target architecture is the “InstructionEncoding” TableGen class defined in llvm/include/llvm/Target/Target.td. This class holds the decoder method and size of instruction in addition to minor variables. It gets inherited to the generic “Instruction” class which is defined in the same class. This class holds input and output DAGs and information which is useful to the compiler and is generalizable to all architectures.
\par

The general class gets inherited to every target-specific class. In RISC-V’s case, the next stop of the instruction is the “RVInst” class which inherits from the general “Instruction” class and it resides in llvm/lib/Target/RISCV/RISCVInstrFormats.td TableGen file. It defines the general bit patterns of RISC-V instructions. For example, the opcode being the first 7 bits. It defines additional information like the assembly string pattern. This general class is inherited by every type of instruction of R, I, S, B, U, and J types. As a simple example, XOR instruction can be traced. As XOR is an R type, a register-register instruction, it continues its inheritance journey from “RVInstR”. It is common to R type instructions to have funct7, rs2, rs1, funct3, and rd format ordered from MSB to LSB. These variables are assigned corresponding bit fields in the class. 
\par

The RISC-V formats mentioned are included in the llvm/lib/Target/RISCV/RISCVInstrInfo.td file which is in the same directory as the RISCVInstrFormats.td file. After inclusion, the “RVInstR” class gets inherited by the “ALU\_rr” class. The “ALU\_rr” class adds the commutability feature which means swapping source 1 and source 2 does not create a different result like in addition but not in subtraction. In the end, XOR’s record is defined by putting funct7, funct3 and assembly string manually in a single line with scheduling information added. 

\section{Adding a New Instruction Using TableGen}\label{sec:MLA_add_section}
We created a tutorial to use as a reference while adding more complex instructions.
Create a new tablegen file for custom additions and include it at the end of the RISCVInstrInfo.td file. We named it RISCVInstrInfoCrypt.td as it is going to be cryptography related.
\begin{lstlisting}[caption= Include file]
include "RISCVInstrInfoCrypt.td"
\end{lstlisting}

Add the specifications of the instruction to the RISCVInstrInfoCrypt.td file. Here we created a new class of instruction is defined named ALU\_rrr. For the MLA instruction, the specifications are:

\begin{lstlisting}
let hasSideEffects = 0, mayLoad = 0, mayStore = 0 in
class ALU\_rrr<bits<2> funct2, bits<3> funct3, string opcodestr,
            bit Commutable = 0>
   : RVInstR4<funct2, funct3, OPC_OP, 
   (outs GPR:$rd), (ins GPR:$rs1, GPR:$rs2, GPR:$rs3),
             opcodestr, "$rd, $rs1, $rs2 ,$rs3"> {
 let isCommutable = Commutable;
}
\end{lstlisting}
Explanation of ALU\_rrr:
\begin{lstlisting}
let hasSideEffects = 0, mayLoad = 0, mayStore = 0 in
\end{lstlisting}

If the instruction has no side effect, hasSideEffects will be 0
If there is no need or possibility to load data from memory, mayLoad will be 0
If there is no need or possibility to store data from memory, mayStore will be 0


\begin{lstlisting}
class ALU_rrr<bits<2> funct2, bits<3> funct3, string opcodestr,
            bit Commutable = 0>
\end{lstlisting}

Class is defined with ALU\_rrr name. Variables are defined. funct2 is two bits of binary number as RVInstR4 is used which reserves 5 bits of funct7 for another register. funct3 is three bits of binary number. opcodestr is the string that will be shown in the assembly file. Commutable is one bit of zero which determines the importance of the order of the inputs.


\begin{lstlisting}
: RVInstR4<funct2, funct3, OPC_OP, (outs GPR:$rd), (ins GPR:$rs1, GPR:$rs2, GPR:$rs3),
\end{lstlisting}

RVInstR4 instruction type is called from RISCVInstrFormats.td file. funct2, funct3, opcode, output, and inputs are entered in function orderly.

\begin{lstlisting}
opcodestr, "$rd, $rs1, $rs2 ,$rs3"> {
 let isCommutable = Commutable;
}
\end{lstlisting}

opcode string and activating commutability option.

Create another file in which we are going to add the schedules. In our case, the name of this file is “RISCVInstrInfoCrypt.td”. The definition of the instruction should be added to this file by using ALU\_rrr class defined above and inputting the correct scheduling variables. For the MLA instruction, it is:

\begin{lstlisting}
def MLA : ALU_rrr<0b10, 0b100, "mla">,
Sched<[WriteIMul, ReadIMul, ReadIMul]>;
\end{lstlisting}

MLA instruction is defined and ALU\_rrr instruction type is used. funct2,funct3, and opcode strings are entered. Schedules are determined.

Add the instruction’s pattern defining its logic to the RISCVInstrInfoCrypt.td file. For the MLA instruction, it is:

\begin{lstlisting}
def : Pat< (add (mul GPR:$src1, GPR:$src2), GPR:$src3),
(MLA GPR:$src1, GPR:$src2, GPR:$src3)>;
\end{lstlisting}

MLA instruction is called from RISCVInstrInfoCrypt.td file and inputs and outputs are determined.

\section{Adding Pattern Matching Support for New Instruction Using C++}
TableGen aims to provide a declarative way to introduce new patterns for new instruction developers. However, not all instructions can be described in this scheme.  Although it is called "dag" as a keyword in TableGen, it expects a tree of instructions. Custom C++ can be the only way to match until the TableGen based system improves. It is also possible to use C++ in complex patterns in TableGen, which can make the most of the pattern declarative and only the necessary part in imperative style.
\par
A domain that TableGen fails is matching a graph of instructions with dependant operands. As an example we can think of an instruction having two operands of Load Instructions. If the Load instructions are from an array, they must be related to each other by an offset and it might need to be detected for certain patterns.


\lstinputlisting[caption={Minimal Subtree of Optimized S-box LLVM IR},linerange={1-5} ,label={lst:sbox-xor},language=llvm,style=nasm]{../s-box/keccakO3.ll}

In Code \ref{lst:sbox-xor}, it can be seen that the XOR instruction is between the first element of the input struct and the fifth element of it. As the locations of elements matters, they must be matched by not only looking at Load instructions but also their operands. What we are looking for is to have the base of Load instruction to be equal and the offset operand of it to evaluate to 4, designating the fifth element of struct. TableGen is not suitable for this operation and the source code of SelectionDAG should be analyzed to place the logic to pattern match this set of instructions.

The process for adding an instruction via C++ is as follows:
\begin{enumerate}
    \item Create a record declaration in TableGen to provide the Assembler support, ignoring the Pattern declaration.
    \item Observe the DAG in the debug output or dot file and locate the root of it. 
    \item Add a function in RISCVISelDAGToDAG.cpp file in the root instruction case. 
    \item Implement pattern matching and replacement with SelectionDAG node.
\end{enumerate}

%TODO what is a basic block, maybe in IR section 
SelectionDAG consists of numerous files but the most relevant ones to the developer can be few if the complexity of the pattern is small. The order of files will be from IR to Assembly. RISCVCodeGenPrepare.cpp file provides mechanisms for matching in IR form. This file exists mainly due to the limitation of SelectionDAG which is running per basic-block. RISCVISelLowering.cpp contains the lowering of IR to SelectionDAG nodes. It can decide on whether a type or expression should be legalized or expanded. RISCVISelDAGToDAG.cpp is the instruction selector in C++. Its implementation traverses the DAG from the root and runs the selection functions depending on the SelectionDAG node type which is parallel to Opcodes. 

%TODO C++ code for matching will be added




%TODO Move to Appendix
\section{Creating Assembly File From C File}

LLVM consists of many flexible libraries which allows user to use different libraries with preffered  options. To create RISC-V assembly from c code, Clang and LLC are used with the following commands. \\

Clang is the C compiler front-end which is mainly used with LLVM backend. Clang is used in this project to produce LLVM intermediate representation code. Following command produces a .ll file in the current directory. 

\begin{lstlisting}[language=Bash]
-clang -S -target riscv32-linux-gnu -emit-llvm -g foo.c
\end{lstlisting}

-S option provides to run only preprocess and compilation steps. \\
-target option specifies the 32 bit RISC-V target architecture. \\
-emit-llvm is for targeting the LLVM backend. \\
-g option generates the source level debug information.\\

LLC is the LLVM compiler back-end which converts LLVM intermediate representation (IR) into native machine code for a specific target architecture. Following command produces a .s file for RISC-V architecture in the current directory. 

\begin{lstlisting}[language=Bash]
llvm-project/build/bin/llc -debug-only=isel -view-isel-dags -mtriple=riscv32 lxr.ll
\end{lstlisting}
%TODO lets set standard bin in this thesis, similar todo exists 

-debug-only=isel option gives the debug informations during the DAG lowering process.\\
-view-isel-dags option prints the directed acylic graph (DAG) image of the IR code.\\
-mtriple=riscv32 defines the 32 bits RISC-V target architecture.\\

%%%%%%%%%%%%%%%%%%%%%%%%%%%%%%%%%%%%%%%%%%%%%%%%%%%%%%%%%%%%%%%%%
% !TEX root = interimreport.tex
\clearpage
\chapter{UPCOMING WORK}\label{Ch6}
%%%%%%%%%%%%%%%%%%%%%%%%%%%%%%%%%%%%%%%%%%%%%%%%%%%%%%%%%%%%%%%%%
\section{Instructions to be Added}
After experimenting with MLA instruction, we were given a new hardware target which lacked support in LLVM \cite{eryilmaz}. The hardware is based on the RISC-V instruction set extended by rotation (ROT) and s-box which is specific to the ASCON encryption algorithm the hardware aims to accelerate. 
\par
\subsection{ROT/ROTI Instruction}
Due to complications in LLVM at representing circular shift instead of linear shift at IR level, a new intrinsic function funnel shift (llvm.fshl) was introduced to the codebase. To see if it is generated at the assembly level, it is a better practice to declare the function in the .ll file that is fed into LLC program. As ROT is a more popular instruction to be extended LLVM may have support already through for example the bit manipulation extension and moreover the ways to enable it will be investigated. 

\subsection{S-BOX Instruction}
 
\begin{figure}
    \centering
    \includegraphics{upcoming_work/s_box_ascon.png}
    \caption{S-Box Layer of ASCON}
    \label{fig:sbox}
\end{figure}
S-Box instruction used in the ASCON algorithm is a series of XORs and ORs between bits of 5 words which can be seen in Figure \ref{fig:sbox}. The instruction is hard to generalize and the hardware implementation is implementing the instruction as from memory to memory which poses another set of challenges for us to overcome.
\section{Hardware Modifications}
The ROT instruction is implemented on hardware however the immediate version of it, ROTI, is not. After implementing the ROT instruction at LLVM we will try to modify the hardware without harming its integrity.

%%%%%%%%%%%%%%%%%%%%%%%%%%%%%%%%%%%%%%%%%%%%%%%%%%%%%%%%%%%%%%%%%
%\chapter{GANTT CHART / TIME PLAN / WORKFLOW}\label{Ch7}
%%%%%%%%%%%%%%%%%%%%%%%%%%%%%%%%%%%%%%%%%%%%%%%%%%%%%%%%%%%%%%%%%


%%%%%%%%%%%%%%%%%%%%%%%%%%%%%%%%%%%%%%%%%%%%%%%%%%%%%%%%%%%%%%%%%
%\chapter{PROJECT INFORMATION IF THE THESIS IS BASED ON A PROJECT (TUBITAK, SANTEZ, EU PROJECTS ETC.)}\label{Ch8}
%%%%%%%%%%%%%%%%%%%%%%%%%%%%%%%%%%%%%%%%%%%%%%%%%%%%%%%%%%%%%%%%%


% ----------------------------------------------------------------- %
% Form thesis_bib.bib to contain your references in bibtex format   %
% use \cite command for citing your references.                     %
% ----------------------------------------------------------------- %

\bibliographystyle{thesis_itubib}      % Designed .bst file 
\bibliography{thesis_bib}			   % .bib file

% ----------------------------------------------------------------- %
% Appendix files. appendices_cover is the appendix title page.      %   
% If Appendix not available, comment all the related commands below %
% ----------------------------------------------------------------- %

\eklerkapak{}
%\vglue18pt  % This has to be 18 pt - SBÖ
\singlespacing


\begin{itemize}[leftmargin=3.3cm,itemsep=-0.4em,labelsep=1.5mm] % Adust margin to flush left, item sep., label sep. - SBÖ
\item [\textbf{APPENDIX A.1 :}]Installation of Software

\end{itemize}

\newpage

% ----------------------------------------------------------------- %
% \eklerbolum{x} forms and appendix of x chapters. 				    %
% ----------------------------------------------------------------- %

\eklerbolum{0}
\chapter{APPENDIX A.1}
\vglue6pt
% For Appendix A.1
% Format the equation environment
\renewcommand{\theequation}{A.1.\arabic{equation}}
% Reset the counter
\setcounter{equation}{0}
\section{Installation of Software}

As the LLVM codebase is large and has many options while building from source, finding the right options that our computers can handle easily was both essential to get started and critical as it decides the time it takes to see a change in code to get compiled. For this purpose, we accumulated the commands and created a tutorial that we can use in the future.

\begin{lstlisting}[language=Bash, caption={Clone Repository and Install Necessary Packages}]
git clone https://github.com/llvm/llvm-project
cd llvm-project
mkdir build
cd build	
sudo apt install cmake, ninja-build, clang, lld	
\end{lstlisting}

\begin{lstlisting}[language=Bash, caption={CMake Configuration We Used}]
cmake -S ../llvm . -G Ninja -DCMAKE_BUILD_TYPE="Debug"  \
-DBUILD_SHARED_LIBS=True -DLLVM_USE_SPLIT_DWARF=True  \
-DLLVM_BUILD_TESTS=True   -DCMAKE_C_COMPILER=clang \
-DCMAKE_CXX_COMPILER=clang++ -DLLVM_TARGETS_TO_BUILD="all" \
-DLLVM_EXPERIMENTAL_TARGETS_TO_BUILD="RISCV" -DLLVM_ENABLE_LLD=ON	
\end{lstlisting}
With this command, we are choosing the type as debug. Shared\_libs=TRUE causes all libraries to be built shared instead of static libraries. ..SPLIT\_DWARF is set to True to minimize memory usage at link time. We want to use clang as the C compiler. Therefore, it is specified in the command as DCMAKE\_C\_COMPILER=clang. In addition to that, we want to use lld as the linker instead of gold, so we specify that as well.
This configuration is the most efficient in terms of memory and disk usage among our previous attempts at building LLVM from source.
\noindent
\begin{minipage}[t]{0.35\linewidth}
\begin{lstlisting}[language=Bash, caption={To build from scratch or to rebuild files with change, automatically}]
Ninja
\end{lstlisting}
\end{minipage}
% \hspace{2em}
\begin{minipage}[t]{0.25\linewidth}
\hfill
\end{minipage}
\begin{minipage}[t]{0.35\linewidth}
\begin{lstlisting}[language=Bash, caption={To build llc only which is the binary we modify}]
ninja llc	
\end{lstlisting}
\end{minipage}



While running ninja, CPU and ram usage significantly increases. All available cores are used capacity. This may prevent doing other tasks while running ninja. In order to prevent this one may opt to use the following command instead. It allows you to choose how many cores are going to be utilized.
\begin{lstlisting}[language=Bash]
	ninja llc -j<number of cores to use>
\end{lstlisting}


\section*{APPENDIX A.2}
\vglue6pt
% For Appendix A.2
% Format the equation environment
\renewcommand{\theequation}{A.2.\arabic{equation}}
% Reset the counter
\setcounter{equation}{0}
The output of the unoptimised S-box function is provided below. The C code used to produce this LLVM IR is in Code \ref{lst:sbox-c}

\lstinputlisting[language=llvm,style=nasm]{../s-box/opt/unoptimised.ll}

\newpage



% ----------------------------------------------------------------- %
% Additional pages can be added to the below section                %
% ----------------------------------------------------------------- %

\end{document}                      

% ----------------------------------------------------------------- %
% About this template: iletisim@be.itu.edu.tr                       %
% Updated by	:													%
% E. Baris Ondes: ondes@itu.edu.tr 	 OR ondesalt@pm.me				% 
% Tamer Sener	: senerta@itu.edu.tr OR tamsener@gmail.com			%
% Berkan Kacmaz	: kacmazb@itu.edu.tr OR kacmazberkan0@gmail.com		%
% S. Baris Ozturk : ozturksb@itu.edu.tr OR salihbaris@gmail.com		%
% ----------------------------------------------------------------- %
